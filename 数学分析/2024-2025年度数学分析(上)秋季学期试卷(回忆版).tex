\subsection{2024-2025年度数学分析(上)秋季学期试卷(回忆版)}

\subsubsection*{一、判断题(10道,每题2分)}

\begin{enumerate}
    \item 有上界的非空数集其上界集合必有最小值。\hfill (T)
    \item 有理数和无理数一样多,交替出现。\hfill (F)
    \item 若\(a_n \le b_n \le c_n\),且\(\{a_n\},\{c_n\}\)均收敛,则\(b_n\)也收敛。\hfill (F)
    \item \(\sigma\)的读音是delta。\hfill (F)
    \item 函数在某一点处可微,则在各个方向上都有方向导数。\hfill (T)
    \item 闭区域一定为连通集。\hfill (T)
    \item 一致连续函数必有界。\hfill (F)
    \item 拐点处可能存在极值。\hfill (T)
    \item 聚点一定不是外点。\hfill (T)
    \item 有界闭集覆盖\(\Rightarrow\)存在有限子覆盖。\hfill (F)
\end{enumerate}

\subsubsection*{二、计算题(2题,每题10分)}

\begin{enumerate}
    \item (10分) 设\(y=(u+v,u-v,u^2v),x=(u,v)\)求\(\frac{dy}{dx}\)。

    \item (10分) 求椭圆\(x^2+3y^2=12\)的内接等腰三角形,其底边平行于椭圆的长轴,而使面积最大。
\end{enumerate}

\subsubsection*{三、证明题(6题,每题10分)}

\begin{enumerate}
    \item (10分) 证明:\(\sqrt{2}+\sqrt{5}\)不是有理数。

    \item (10分) 证明:欧式空间中的紧集等价于有界闭集。

    \item (10分) 设\(f(x)\)在\((0,+\infty)\)上连续,且满足\(f(x^2)=f(x), x \in(0,+\infty)\),证明\(f(x)\)在\((0,+\infty)\)上为常数函数。

    \item (10分) 证明:欧式空间中,定义域为紧集的连续函数为一致连续函数。

    \item (10分) 已知\(\lim_{n \rightarrow \infty} a_n=a, \lim_{n \rightarrow \infty} b_n=b\),证明:
    \begin{equation*}
        \lim_{n \rightarrow \infty} \frac{a_1 b_n+a_2 b_{n-1}+\cdots+a_n b_1}{n}=ab.
    \end{equation*}

    \item (10分) 证明:函数\(f(x)=\sin(x^2)\)在区间\([0,1]\)上一致连续,在\(\mathbb{R}\)上不一致连续。
\end{enumerate}