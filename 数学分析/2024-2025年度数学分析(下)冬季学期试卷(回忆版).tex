\subsection{2024-2025年度数学分析(下)冬季学期试卷(回忆版)}

\subsubsection*{一、判断题(10道题,一题2分)}

\begin{enumerate}
    \item \(r=a \mathrm{e}^{\theta}\)为阿基米德螺线。\hfill (F)
    \item 数列必有无穷多项小于等于其上极限。\hfill (T)
    \item 级数若绝对收敛,则必条件收敛。\hfill (F)
    \item \(\psi\)的读音是"佛爱 phi"。\hfill (F)
    \item 无穷乘积收敛,则其通项极限为1。\hfill (T)
    \item 若函数黎曼可积,则必有界。\hfill (T)
\end{enumerate}

\subsubsection*{二、计算题(4道,每题10分)}

\begin{enumerate}
    \item (10分) 求\(\mathrm{d} \omega\).(每题五分)
    \begin{enumerate}
        \item[(1)] \(R \mathrm{d}x \mathrm{d}y+P \mathrm{d}y \mathrm{d}z+ Q \mathrm{d}z \mathrm{d}x\)
        \item[(2)] \(P \mathrm{d}x +Q \mathrm{d}y + R \mathrm{d}z\)
    \end{enumerate}

    \item (10分) 判断\(\sum_{i=1}^{+\infty}(1-x)x^{n}\)在区间[0,1]的一致收敛性。

    \item (10分) 求\(\iint_D \sqrt{\frac{1+x^2+y^2}{1-x^2-y^2}}\),其中\(D\)为圆\(x^2+y^2=1\)在第一象限的部分。

    \item (10分) 求\(\int_0^{+\infty}\frac{\sin x}{x}\)。
\end{enumerate}

\subsubsection*{三、证明题(4道,每题10分)}

\begin{enumerate}
    \item (10分) 对任意在\([a,b]\)上有界的函数\(f(x)\),恒有
    \begin{equation*}
        \lim_{\lambda\to 0} \bar{S}(P)=L.
    \end{equation*}

    \item (10分) 设\(\psi(x)\)在[0,\(+\infty\)]上连续且单调,\(\lim_{x \to \infty}\psi(x)=0\),证明
    \begin{equation*}
        \lim_{p \to \infty}\int_0^{+\infty}\psi(x)\sin px \mathrm{d} x=0.
    \end{equation*}

    \item (10分) 设函数\(f\)在\([a,b]\)上单调上升且非负,函数\(g(x)\)在[a,b]上可积,则存在\(c \in [a,b]\)使得
    \begin{equation*}
        \int_a^b f(x)g(x) \mathrm{d}x=f(b)\int_c^b g(x) \mathrm{d}x.
    \end{equation*}

    \item (10分) 证明\(\frac{x \mathrm{d}x+y\mathrm{d}y}{x^2+y^2}\)为某个二元函数的全微分,并求出所有的这种二元函数。
\end{enumerate}