\subsection{Assignment-2}

\begin{problem}
    设曲面$S$由$(r\cos (\theta),r \sin (\theta), 2 \theta)$, 其中$0 \le r \le 4,0 \le \theta \le 2 \pi$.\\
    (1) 求曲面$S$在点$\displaystyle (\sqrt{2},-\sqrt{2},\frac{7 \pi}{2})$处的切平面方程$T_pS$;\\
    (2) 求曲面$S$的面积元$\mathrm{d}A$;\\
    (3) 设曲线$\gamma:(2 \cos (\theta), 2 \sin (\theta), 2 \theta)$, 求该曲线在$\displaystyle \theta= \frac{7 \pi}{4}$处的点$P_0$的曲率$k$和挠率$\tau$;\\
    (4) 求曲面在$P_0$处的法曲率$k_n$.
\end{problem}

\begin{solution}
    (1) $\overrightarrow{r}=(r\cos (\theta),r \sin (\theta), 2 \theta)$,求导有
    \begin{align*}
        \overrightarrow{r}_r=\begin{bmatrix}
            \cos(\theta)\\ \sin(\theta)\\ 0
        \end{bmatrix},
        \overrightarrow{r}_\theta=\begin{bmatrix}
            -r\sin(\theta)\\ r\cos(\theta)\\ 2
        \end{bmatrix}.
    \end{align*}
    因此$S$在点$\displaystyle (\sqrt{2},-\sqrt{2},\frac{7 \pi}{2})$,即$\displaystyle \overrightarrow{r}(2,\frac{7 \pi}{4})$处的切平面方程为
    \begin{align*}
        0=\det \begin{bmatrix} \displaystyle
            x-\sqrt{2} &y+\sqrt{2} &z-\frac{7 \pi}{2}\\
            \frac{\sqrt{2}}{2} &-\frac{\sqrt{2}}{2} &0 \\
            \sqrt{2} &\sqrt{2} &2
        \end{bmatrix},
    \end{align*}
    即$\sqrt{2} x+\sqrt{2} y-2 z+7 \pi =0$.\\
    (2) $\mathrm{d}A=|\overrightarrow{r}_r \times \overrightarrow{r}_\theta | \mathrm{d}r \mathrm{d}\theta=\sqrt{r ^2+4 }\mathrm{d}r \mathrm{d}\theta $.\\
    (3) $\overrightarrow{r}'=(-2 \sin(\theta),2\cos(\theta),2),\overrightarrow{r}''=(-2\cos(\theta),-2\sin(\theta),0),\overrightarrow{r}'''=(2\sin(\theta),-2\cos(\theta),0)$,故
    \begin{align*}
        k=\frac{|\overrightarrow{r}' \times \overrightarrow{r}''|}{|\overrightarrow{r}'|^3}= \frac{1}{4},\\
        \tau=\frac{|(\overrightarrow{r}',\overrightarrow{r}'',\overrightarrow{r}''')|}{|\overrightarrow{r}' \times \overrightarrow{r}''|^2}=\frac{1}{4}.
    \end{align*}
    (4) 
    \begin{gather*}
        \overrightarrow{n}=\frac{\overrightarrow{r}_r \times \overrightarrow{r}_\theta}{|\overrightarrow{r}_r \times \overrightarrow{r}_\theta|}=(\frac{2\sin(\theta)}{\sqrt{r^2+4}},-\frac{2\cos(\theta)}{\sqrt{r^2+4}},\frac{r}{\sqrt{r^2+4}}), \\ \overrightarrow{n}_r=(-\frac{2r\sin(\theta)}{(r^2+4)^{\frac{3}{2}}},\frac{2r\cos(\theta)}{(r^2+4)^{\frac{3}{2}}},\frac{4}{(r^2+4)^{\frac{3}{2}}}), \\ \overrightarrow{n}_\theta=(\frac{2\cos(\theta)}{\sqrt{r^2+4}},\frac{2\sin(\theta)}{\sqrt{r^2+4}},0).
    \end{gather*}
    \begin{align*}
        M_p &= -\begin{bmatrix}
            u &v 
        \end{bmatrix}\begin{bmatrix}
            \overrightarrow{r}_r \cdot \overrightarrow{n}_r &\overrightarrow{r}_r \cdot \overrightarrow{n}_\theta \\
            \overrightarrow{r}_\theta \cdot \overrightarrow{n}_r &\overrightarrow{r}_\theta \cdot \overrightarrow{n}_\theta        \end{bmatrix}\begin{bmatrix}
                u \\ v
            \end{bmatrix}\\
        &= -\frac{4 u v}{\sqrt{r^2+4}}.
    \end{align*},
    \begin{align*}
        D_p &= \begin{bmatrix}
            u &v 
        \end{bmatrix}\begin{bmatrix}
            \overrightarrow{r}_r \cdot \overrightarrow{r}_r &\overrightarrow{r}_r \cdot \overrightarrow{r}_\theta \\
            \overrightarrow{r}_\theta \cdot \overrightarrow{r}_r &\overrightarrow{r}_\theta \cdot \overrightarrow{r}_\theta        \end{bmatrix}\begin{bmatrix}
                u \\ v
            \end{bmatrix}\\
        &= \left(r^2+4\right) v^2+u^2.
    \end{align*},
    \begin{align*}
        k_n &=\frac{M_p}{D_p} \\
        &= -\frac{4 u v}{\sqrt{r^2+4} \left(\left(r^2+4\right) v^2+u^2\right)}.
    \end{align*}
    代入$r=2,\theta=\displaystyle \frac{7\pi}{4}$,有
    \begin{align*}
        k_n =-\frac{\sqrt{2} u v}{u^2+8 v^2}.
    \end{align*}
\end{solution}

\begin{problem}{}{}
    求准线为$\begin{cases}
        z=\sqrt{x^2+y^2} \\
        x=-1
    \end{cases}$,母线$\mathbb{R} \overrightarrow{V}$的柱面方程,其中$\overrightarrow{V}=(1,1,1)$.
\end{problem}

\begin{solution}
    $\begin{cases}
        \frac{x-a}{1}=\frac{y-b}{1}=\frac{z-c}{1} \\
        c=\sqrt{a^2+b^2} \\
        a=-1
    \end{cases}
    \Rightarrow z=1+x+\sqrt{1+(y-x-1)^2}$
\end{solution}

\begin{problem}{}{}
    曲面方程由$x^4+y^4+3x^2z^2+7xy^2z+2yz^3=0$给出,判断曲面的形状;并证明曲面是直纹面.
\end{problem}

\begin{solution}
    因为$F(tx,ty,tz)=t^4F(x,y,z)$,因此该方程为齐次方程,由Theorem 4.6知其为锥面,又由直纹面定义可知,锥面为直纹面,因此该曲面的形状为锥面,且为直纹面。
\end{solution}