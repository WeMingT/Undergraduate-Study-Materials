\subsection{Assignment-3}

\begin{problem}
    求准线为$\begin{cases}
         z=\sqrt{x^2+y^2}&\text{}\\
         z=y+1&\text{}\\
    \end{cases} $,顶点为$(1,0,0)$的锥面方程.
\end{problem}

\begin{solution}
    $\begin{cases}
        \displaystyle \frac{x-1}{x_1-1}=\frac{y}{y_1}=\frac{z}{z_1}\\
        z_1=\sqrt{x_1^2+y_1^2}\\
        z_1=y_1+1
    \end{cases}
    \Rightarrow \Gamma:x^2+2y^2-2xy-2yz+2zx-2x+2y-2z+1=0.$
\end{solution}

\begin{problem}
    设曲线$\Gamma:4x^2-4xy+y^2-6x-8y+3=0$.\\
    (1) 判别曲线是否是有心曲线,并给出理由.\\
    (2) 求出曲线$\Gamma$的渐近方向,是否有渐近线?\\
    (3) 求主直径与主方向.\\
    (4) 化简曲线,并判断其类型.
\end{problem}

\begin{solution}
    (1) $I_2=|A|=\det \begin{bmatrix}
        4 &-2\\
        -2 &1
    \end{bmatrix}=0 \Rightarrow \Gamma$不是有心曲线.\\
    (2) $\mathbf{v}^\prime A \mathbf{v}=0 \Rightarrow \mathbf{v}=\begin{bmatrix}
        1\\2
    \end{bmatrix}\Rightarrow $渐近方向:$\mathbf{v}=\begin{bmatrix}
        1\\2
    \end{bmatrix}$.\\
    因为$\Gamma$为无心曲线,因此$\Gamma$无渐近线.\\
    (3) $(A-\lambda I)\mathbf{v}=0 \Rightarrow \mathbf{v}_1=\begin{bmatrix}
        1\\ 2
    \end{bmatrix},\mathbf{v}_2=\begin{bmatrix}
        -2\\1
    \end{bmatrix} \Rightarrow \Gamma$的主方向:$\mathbf{v}_1=\begin{bmatrix}
        1\\2
    \end{bmatrix},\mathbf{v}_2=\begin{bmatrix}
        -2\\1
    \end{bmatrix}.$\\
    $\mathbf{v}^\prime(A\mathbf{x}+\mathbf{b})=0 \Rightarrow$主直径: $l :-10x+5y+2=0$.\\
    (4) 做转轴变换:$\begin{bmatrix}
        x\\y
    \end{bmatrix}=\begin{bmatrix}
        \frac{1}{\sqrt{5}} &-\frac{2}{\sqrt{5}}\\
        \frac{2}{\sqrt{5}} &\frac{1}{\sqrt{5}}
    \end{bmatrix} \begin{bmatrix}
        x_1\\y_1
    \end{bmatrix}$,有:
    \[\Gamma=\Gamma_1:5\sqrt{5}y_1^2-22x_1+4y_1+3=0,\]
    做移轴变换$\begin{cases}
        x_1=x_2+\frac{71}{110\sqrt{5}}\\
        y_1=y_2-\frac{2}{5\sqrt{5}}
    \end{cases}$,有
    \[\displaystyle \Gamma_1=\Gamma_2:y_2^2=\frac{22}{5\sqrt{5}}x_2,\]
    为抛物线.
\end{solution}

\begin{problem}
    设曲面为$5x^2-16y^2+5z^2+8xy-14xz+8yz+4x+20y+12z=24$.\\
    (1) 求过点$\displaystyle (0,0,\frac{2\sqrt{39}-6}{5})$的切平面.\\
    (2) 求主径面与主方向.\\
    (3) 化简曲面方程,并判断其类型.
\end{problem}

\begin{solution}
    $A=\begin{bmatrix}
        5 &4 &-7\\
        4 &-16 &4\\
        -7 &4 &5
    \end{bmatrix},\mathbf{b}=\begin{bmatrix}
        2\\10\\6
    \end{bmatrix},\mathbf{x_0}=\begin{bmatrix}
        0\\0\\ \displaystyle \frac{2\sqrt{39}-6}{5}
    \end{bmatrix},(A\mathbf{x_0}+\mathbf{b})^\prime (\mathbf{x}-\mathbf{x_0})=0
    \Rightarrow $\\
    切平面: $\pi:(-26+7 \sqrt{39}) x-(13+4 \sqrt{39}) y-5 \sqrt{39} z+78-6 \sqrt{39}=0.$\\
    (2) $(A-\lambda I)\mathbf{v}=0 \Rightarrow$ 主方向: $\mathbf{v}_1=\begin{bmatrix}
        2\\1\\2
    \end{bmatrix},\mathbf{v}_2=\begin{bmatrix}
        -1\\0\\1
    \end{bmatrix},\mathbf{v}_3=\begin{bmatrix}
        1\\-4\\1
    \end{bmatrix}.$\\
    $(A\mathbf{x}+\mathbf{b})^\prime \mathbf{v}=0 \Rightarrow$ 主径面: $\pi_1:-3x+3z+1=0,\pi_2:9x-36y+9z+16=0.$\\
    (3) 做转轴变换$\begin{bmatrix}
        x\\y\\z
    \end{bmatrix}=\begin{bmatrix}
        \frac{2}{3} &-\frac{1}{\sqrt{2}} &\frac{1}{3\sqrt{2}}\\
        \frac{1}{3} &0 &-\frac{4}{3\sqrt{2}}\\
        \frac{2}{3} &\frac{1}{\sqrt{2}} &\frac{1}{3\sqrt{2}}
    \end{bmatrix} \begin{bmatrix}
        x_1\\y_1\\z_1
    \end{bmatrix}$, 有
    \[\Gamma=\Gamma_1:18 y^2-27 z^2+26 x+6 \sqrt{2} y-16 \sqrt{2} z -36=0.\]
    做移轴变换$\begin{bmatrix}
        x_1\\y_1\\z_1
    \end{bmatrix}=\begin{bmatrix}
        x_2+\frac{67}{54}\\y_2-\frac{\sqrt{2}}{6}\\z_2-\frac{8\sqrt{2}}{27}
    \end{bmatrix}$,有
    \[\Gamma_1=\Gamma_2:26 x_2+18 y_2^2-27 z_2^2=0,\]
    为双曲抛物面.
\end{solution}

