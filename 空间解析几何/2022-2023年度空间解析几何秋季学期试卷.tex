\subsection{2022-2023年度空间解析几何秋季学期试卷}

\begin{problem}
    $(20\text{ 分)已知空间中的四个点 }P(1,2,0),Q(2,4,-1),R(5,6,-3),$以及$S(1,1,1)$.\\
(i) 求四个点张出的平行六面体的体积.\\
(ii)用$\Pi_1$表示包含点$P,Q$以及$R$的平面,用$\Pi_2$表示包含点$P,Q$
以及$S$的平面.求$\Pi_1$与$\Pi_2$的夹角.\\
(iii) 求点$S$到平面$\Pi_{1}$的距离.\\
(iv) 求点$S$关于$\Pi_{1}$的反射点.
\end{problem}

\begin{solution}
    
\end{solution}

\begin{problem}
    2.(20分) 曲面$S$ 由 $(r\cos\theta,r\sin\theta,2\theta)(0\leq r\leq4,0\leq\theta\leq$
$2\pi$, $b= 2)$.\\
(a) 求曲面在$\displaystyle (\sqrt2,-\sqrt2,\frac72\pi)$处的切平面方程$T_{P_0}(S)$.\\
(b) 求曲面的面积元$\mathrm{d}A$.\\
(c) 设曲线$\gamma:\left(2\cos\theta,2\sin\theta,2\theta\right)$,求曲线在$\displaystyle \theta=\frac74\pi$处的点$P_0$的曲
率$\mathfrak{k}$以及挠率 $\tau$.\\
(d)求曲面在$P_\mathrm{p}$处的法曲率$\mathfrak{k}_n.$
\end{problem}

\begin{solution}
    
\end{solution}

\begin{problem}
    (20分)\\
(i) 求准线为
$$\begin{cases}z&=\sqrt{x^2+y^2}\\x&=-1\end{cases}$$
母线为$\mathbb{R} \mathbf{v}$ 的柱面方程,其中$\mathbf{v}=(1,1,1)$.\\
(ii) 曲面方程由$x^4+y^4+3x^2z^2+7xy^2z+2yz^3=0$给出.判断曲面
的形状并证明曲面为直纹面.
\end{problem}

\begin{solution}
    
\end{solution}

\begin{problem}
    (20分)设平面曲线由
$$F(x,y)=3x^2+4xy+y^2+2x+4y+6=0$$
给出.\\
(i)求曲线的渐近方向与中心.\\
(ii) 求共轭于(1,1)的直径方程.\\
(iii) 求过点(1,-2)的切线方程.\\
(iv)化简曲线为标准形式.
\end{problem}

\begin{solution}
    
\end{solution}

\begin{problem}
    (20 分)设空间曲面$S$的方程为
$$2x^{2}+10y^{2}-z^{2}+12xy+8yz+12x+4y+8z-1=0$$
(i)求曲面的奇向.\\
(ii) 求曲面上点$(0,0,4+\sqrt{15})$处的切平面方程.\\
(iii) 求曲面的主方向与主径面.\\
(iv)化简曲面方程。
\end{problem}

\begin{solution}
    
\end{solution}