\subsection{2024-2025年度高等代数A(下)冬季学期试卷(回忆版)}

\subsubsection*{填空题}

\begin{enumerate}
    \item 求多项式
    \begin{equation*}
    f(x)=x^{4}+x^{3}-3x^{2}-4x-1 \quad \text{与} \quad g(x)=x^{3}+x^{2}-x-1
    \end{equation*}
    的最大公因式。

    \item 求
    \begin{equation*}
    x^{3}-6x^{2}+15x-14
    \end{equation*}
    的有理根。

    \item 写出二次型对应的矩阵。

    \item 设 \(A\) 为五阶方阵,其不变因子为
    \begin{equation*}
    1,\;1,\;1,\;(\lambda-2)^{2},\;(\lambda-3)(\lambda-1)^{2},
    \end{equation*}
    求 \(A\) 的 Jordan 标准型。

    \item 对 \(\alpha=(\alpha_{1},\beta_{1})\) 与 \(\beta=(\alpha_{2},\beta_{2})\) 定义新的内积
    \begin{equation*}
    (\alpha,\beta)=2\alpha_{1}\alpha_{2}+4\beta_{1}\beta_{2},
    \end{equation*}
    求\(\alpha=(1,1),\beta=(1,-1)\)的度量矩阵。
\end{enumerate}

\subsubsection*{选择题}

\begin{enumerate}
    \item \(p(x)\) 为 \(f(x)\) 的重因式,同时是 \(f'(x)\) 与 \(f(x)\) 的公因式,其条件是 $\underline{}$。
    
    \item 2025 阶实对称矩阵按合同分类有多少种?

    \item 求向量之间的夹角。

    \item 判断矩阵是正定,负定,……。

    \item (题目内容不全)
\end{enumerate}

\subsubsection*{计算题}

\begin{enumerate}
    \item 用非退化线性替换将二次型
    \begin{equation*}
    f(x_{1},x_{2},x_{3},x_{4},x_{5})=x_{1}x_{2}+x_{2}x_{3}+x_{3}x_{4}+x_{4}x_{5}
    \end{equation*}
    化为规范形,并求二次型的正惯性指数和符号差。

    \item 求矩阵
    \begin{equation*}
    \begin{pmatrix}
    1 & -3 & 0  & 3 \\
    -2 & 6 & 0 & 13 \\
    0 & -3 & 1 & 3 \\
    -1 & 2 & 0 & 8
    \end{pmatrix}
    \end{equation*}
    的 Jordan 标准型。

    \item 设 \(\varphi\) 为 \(n\) 维线性空间 \(V\) 上的自同构,若 \(W\) 是 \(\varphi\) 的不变子空间,证明:\(W\) 也是 \(\varphi^{-1}\) 的不变子空间。

    \item (类似)求一正交相似变换矩阵,将矩阵
    \begin{equation*}
    A=\begin{pmatrix}1 & 0 & 2 \\ 0 & 3 & 0 \\ 2 & 0 & 1\end{pmatrix}
    \end{equation*}
    化为对角矩阵。
\end{enumerate}

\subsubsection*{证明题}

\begin{enumerate}
    \item 设 \(f_{1},f_{2},\dots,f_{2024},\; g_{1},g_{2},\dots,g_{2025} \in P[x]\),证明:
    \begin{equation*}
    (f_{1}f_{2}\dots f_{2024},\; g_{1}g_{2}\dots g_{2025})=1
    \end{equation*}
    当且仅当
    \begin{equation*}
    (f_{i},g_{j})=1,\quad i=1,2,\dots,2024;\ j=1,2,\dots,2025.
    \end{equation*}

    \item 证明:若变换 \(T\) 保持内积,即满足
    \begin{equation*}
    (T\alpha, T\beta)=(\alpha,\beta),
    \end{equation*}
    则 \(T\) 为线性变换,从而为正交变换。
\end{enumerate}