\subsection{2024-2025年度高等代数A(上)秋季学期试卷(回忆版)}

\subsubsection*{填空题(5题)}
\begin{enumerate}
    \item 矩阵 $\left(\begin{array}{lll}1 & a & 0 \\ 2 & 1 & 0 \\ 1 & 3 & 1\end{array}\right)$ 不是可逆矩阵, 则 $a$ 的值等于 (\quad).
    \item 设 $\boldsymbol{A}$ 和 $\boldsymbol{B}$ 是 $n$ 阶矩阵, $|\boldsymbol{A}|=2,|\boldsymbol{B}|=-3$, 则 $\left|2 \boldsymbol{A}^* \boldsymbol{B}^{-1}\right|=(\quad)$.
    \item (类似)求$\begin{bmatrix}
        1 &2 &3\\
        2 &3 &0\\
        3 &0 &0
    \end{bmatrix}$的逆矩阵(\quad).
    \item $n$阶实对称矩阵的维数为(\quad).
    \item 已知某3阶矩阵的特征值为$1,2,4$,求其伴随矩阵的迹(\quad).
\end{enumerate}

\subsubsection*{选择题(5题)}
\begin{enumerate}
    \item 包含$V_1,V_2$的最小子空间为(\quad).
\end{enumerate}

\subsubsection*{计算题}
\begin{enumerate}
    \item (类似)计算 $n$ 阶行列式:
    \[
    |\boldsymbol{A}|=\left|\begin{array}{cccccc}
    1 & 2 & 3 & \cdots & n-1 & n \\
    n & 1 & 2 & \cdots & n-2 & n-1 \\
    n-1 & n & 1 & \cdots & n-3 & n-2 \\
    \vdots & \vdots & \vdots & & \vdots & \vdots \\
    3 & 4 & 5 & \cdots & 1 & 2 \\
    2 & 3 & 4 & \cdots & n & 1
    \end{array}\right| .
    \]
    \item (类似)设向量组 $\boldsymbol{\alpha}_1, \boldsymbol{\alpha}_2, \boldsymbol{\alpha}_3$ 线性无关, 向量组 $\boldsymbol{\beta}_1, \boldsymbol{\beta}_2, \boldsymbol{\beta}_3$ 可由 $\boldsymbol{\alpha}_1, \boldsymbol{\alpha}_2, \boldsymbol{\alpha}_3$ 线性表示:
    \[
    \left\{\begin{array}{l}
    \boldsymbol{\beta}_1=\boldsymbol{\alpha}_1+2 \boldsymbol{\alpha}_2+3 \boldsymbol{\alpha}_3, \\
    \boldsymbol{\beta}_2=3 \boldsymbol{\alpha}_1-\boldsymbol{\alpha}_2+4 \boldsymbol{\alpha}_3, \\
    \boldsymbol{\beta}_3=\boldsymbol{\alpha}_2+\boldsymbol{\alpha}_3 .
    \end{array}\right.
    \]
    问 $\boldsymbol{\beta}_1, \boldsymbol{\beta}_2, \boldsymbol{\beta}_3$ 是否线性无关?
    \item (类似)

    1.在四维行向量空间中求从基 $e_1, e_2, \cdots, e_n$ 到 $f_1, f_2, \cdots, f_n$ 的过渡矩阵,其中
    \[
    \begin{aligned}
    & \boldsymbol{e}_1=(1,1,0,1), \boldsymbol{e}_2=(2,1,2,0), \boldsymbol{e}_3=(1,1,0,0), \boldsymbol{e}_4=(0,1,-1,-1) \\
    & \boldsymbol{f}_1=(1,0,0,1), \boldsymbol{f}_2=(0,0,1,-1), \boldsymbol{f}_3=(2,1,0,3), \boldsymbol{f}_4=(-1,0,1,2)
    \end{aligned}
    \]

    2.设线性变换 $\sigma$ 在基 $\varepsilon_1, \varepsilon_2 , \varepsilon_3$ 下的矩阵为 $A=\left(\begin{array}{ccc}1 & 2 & 2 \\ 2 & 1 & 2 \\ 2 & 2 & 1\end{array}\right)$,求$\sigma $的特征值与特征向量,并判断$\sigma $的矩阵是否可以在某一组基下为对角阵.
    \item 求解下列线性方程组, 其中 $k$ 为参数:
    \[
    \left\{\begin{array}{l}
    k x_1+x_2+x_3=-2, \\
    x_1+k x_2+x_3=-2, \\
    x_1+x_2+k x_3=-2 .
    \end{array}\right.
    \]
    \item (类似)设 $\boldsymbol{\alpha}_1=(1,0,-1,0), \boldsymbol{\alpha}_2=(0,1,2,1), \boldsymbol{\alpha}_3=(2,1,0,1)$ 是四维实行向量空间 $V$ 中的向量, 它们生成的子空间为 $V_1$, 又向量 $\boldsymbol{\beta}_1=(-1,1,1,1), \boldsymbol{\beta}_2=$ $(1,-1,-3,-1), \boldsymbol{\beta}_3=(-1,1,-1,1)$ 生成的子空间为 $V_2$, 求子空间 $V_1+V_2$ 和 $V_1 \cap V_2$的基.
\end{enumerate}

\subsubsection*{证明题(1题,8分)}
设 $\boldsymbol{A}, \boldsymbol{B}$ 是 $n$ 阶矩阵, 求证:

\[
\left|\begin{array}{ll}
\boldsymbol{A} & \boldsymbol{B} \\
\boldsymbol{B} & \boldsymbol{A}
\end{array}\right|=|\boldsymbol{A}+\boldsymbol{B}||\boldsymbol{A}-\boldsymbol{B}|
\]

