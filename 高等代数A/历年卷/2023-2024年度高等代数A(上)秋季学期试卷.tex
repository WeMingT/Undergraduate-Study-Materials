\subsection{2023-2024年度高等代数A(上)秋季学期试卷}

\subsubsection*{填空题,共5题,每题3分}

\begin{enumerate}
    \item 设 $\alpha_1=\left(\begin{array}{l}1 \\ 0 \\ 3\end{array}\right) , \alpha_2=\left(\begin{array}{c}-1 \\ 2 \\ 1\end{array}\right) , \alpha_3=\left(\begin{array}{l}1 \\ 1 \\ a\end{array}\right)$ 当满足 \underline{$a \neq 5$} 时,任意三维向量 $\alpha$ 必可由 $\alpha_1, \alpha_2, \alpha_3$ 线性表出.
    \item 设 $A,B$分别为 $m \times n$ 和 $n \times s$ 的矩阵. 若 $r(A B)=n$ .则 $r\left(B^{\text{T}} A^{\text{T}}\right)=\underline{n}$.
    \item 设 $\left(\begin{array}{l}1 \\ 1 \\ 1\end{array}\right)+k_1\left(\begin{array}{l}1 \\ 0 \\ 0\end{array}\right)+k_2\left(\begin{array}{l}0 \\ 1 \\ 0\end{array}\right)$ 是非齐次线性方程组 $A \mathbf{x}=\beta$ 的通解. 其中$k_1,k_2$ 为任意常数. 则 $\left(\begin{array}{l}1 \\ 2 \\ 3\end{array}\right)$ 是 $A x=\underline{3 \beta}$ 的解.
    \item 写出线性方程组$A\mathbf{x}=0,B\mathbf{x}=0$同解的一个充要条件\underline{$\operatorname{r}\begin{bmatrix}
        A
    \end{bmatrix}=\operatorname{r}\begin{bmatrix}
        A\\B
    \end{bmatrix}=\operatorname{r}\begin{bmatrix}
        B
    \end{bmatrix}.$}
    \item $r(A+B) \underline{\leq} r(A)+r(B)$.
\end{enumerate}

\subsubsection*{选择题,共5题,每题3分}

\begin{enumerate}
    \item 当 $n \geqslant 2$ 时, 求 $\left|\begin{array}{ccccc}1 & 2 & 2 & \cdots & 2 \\ 2 & 2 & 2 & \cdots & 2 \\ \vdots & \vdots & \vdots & & \vdots \\ 2 & 2 & 2 & \cdots & n\end{array}\right|$ ,则结果是(C).
    \begin{enumerate}
        \item[(A)] $-2 n!$ 
        \item[(B)] $-2(n+2)!$
        \item[(C)] $-2(n-2)!$
        \item[(D)]  0
    \end{enumerate}
    \item $n$ 阶矩阵具有$n$个不同的特征值是$A$与对角阵相似的(B)条件.
    \begin{enumerate}
        \item[(A)] 充要
        \item[(B)] 充分不必要
        \item[(C)] 必要不充分
        \item[(D)] 不充分与不必要
    \end{enumerate}
    \item 设向量组$A:\alpha_1,\alpha_2,\dots,\alpha_s$可由向量组$B:\beta_1,\beta_2,\dots,\beta_t$线性表示,则(A).
    \begin{enumerate}
        \item[(A)] 若$s>t$,则向量组$A$线性相关
        \item[(B)] 若$s<t$,则向量组$A$线性无关
        \item[(C)] 若$s>t$,则向量组$A$线性无关
        \item[(D)] 若$s<t$,则向量组$B$线性相关
    \end{enumerate}
    \item 设 $\alpha_1, \alpha_2$ ,$\alpha_3$ 是线性空间 $V$的一组基则( D )是$V$的另一组基.
    \begin{enumerate}
        \item[(A)] $\alpha_1+\alpha_2+\alpha_3$
        \item[(B)] $\alpha_1-\alpha_2, \alpha_2-\alpha_3$
        \item[(C)] $\alpha_1, \alpha_1-\alpha_2+\alpha_3, \alpha_3-\alpha_2$
        \item[(D)] $\alpha_1+\alpha_2, \alpha_2+\alpha_3, \alpha_3+\alpha_1$
    \end{enumerate}
    \item 设 $A,B$相抵,且 $A$ 有一个$k$ 阶子式不等于 0 ,则 $r(B)$ \underline{B} $k$.
    \begin{enumerate}
        \item[(A)] =
        \item[(B)] $\ge$
        \item[(C)] >
        \item[(D)] $\le$
    \end{enumerate}
\end{enumerate}

\subsubsection*{计算题,4题,共54分}

\begin{enumerate}
    \item 求$\left|\begin{array}{llll}a^2 & (a+1)^2 & (a-2)^2 & (a+3)^2 \\ b^2 & (b+1)^2 & (b-2)^2 & (b+3)^2 \\ c^2 & (c+1)^2 & (c-2)^2 & (c+3)^2 \\ d^2 & (d+1)^2 & (d-2)^2 & (d+3)^2\end{array}\right|$.
    \begin{solution}
        0.
    \end{solution}
    \item 设线性变换 $\sigma$ 在基 $\varepsilon_1, \varepsilon_2 , \varepsilon_3$ 下的矩阵为 $A=\left(\begin{array}{ccc}1 & 2 & 2 \\ 2 & 1 & 2 \\ 2 & 2 & 1\end{array}\right)$,求$\sigma $的特征值与特征向量,并判断$\sigma $的矩阵是否可以在某一组基下为对角阵.
    \item 设有线性方程组$\begin{cases}
        x_1+\lambda x_2+\mu x_3+x_4=0\\ 2x_1+x_2+x_3+2x_4=0 \\ 3x_1+(2+\lambda)x_2+(4+\mu) x_3+ 4x_4=1
    \end{cases}$,已知$\begin{pmatrix}
        1\\-1\\1\\-1
    \end{pmatrix}$是该方程组的一个解,试求:
    \begin{enumerate}
        \item[(1)] 该方程组的全部解,并用对应的齐次线性方程组的基础解系表示全部解.\\
        \item[(2)] 该方程组满足$x_2=x_3$的全部解.
    \end{enumerate}
    \item 数域$F$上$n$阶反对称矩阵全体$V=\{A\in F^{n \times n};A^\prime=-A\}$,按照通常的矩阵加法和数乘构成$F$上的线性空间,请给出它的一组基.
\end{enumerate}

\subsubsection*{证明题,2题,每题8分}

\begin{enumerate}
    \item 设A是 $n$ 阶方阵.则 $|A|=0$ 的充要条件是存在非零矩阵 $B$, 使 $A B=0$.
    \item 设$A,B$为别为$m \times n, n \times s$矩阵,求证$F^n$的子空间$W=\{Bx;ABx=0\}$的维数等于 $r(B)-r(A B)$.
    \begin{proof}
        将矩阵同构于线性映射: $U \stackrel{T}{\longrightarrow} V \stackrel{S}{\longrightarrow} W$.
    其中$T$与$B$同构,$S$与$A$同构.\\
    则有 $W=\operatorname{ker}(S) \cap \operatorname{im}(T)$,\\
    而$$
    \begin{aligned}
    \operatorname{r}(S T) & =\operatorname{dim}\left(\operatorname{im}\left(S |_{\operatorname{im} {(T)}}\right)\right) \\
    & =\operatorname{dim}(\operatorname{im}(T))-\operatorname{dim}(\operatorname{ker}(S |_{\operatorname{im} (T)})) \\
    & =\operatorname{dim}(\operatorname{im}(T))-\operatorname{dim}(\operatorname{ker}(S) \cap \operatorname{im}(T)) \\
    & =\operatorname{r}(T)-\operatorname{dim}(W),
    \end{aligned}
    $$
    故 $\operatorname{dim}(W)=\operatorname{r}(T)-\operatorname{r}(S T)=\operatorname{r}(B)-\operatorname{r}(A B)$.
    \end{proof}
\end{enumerate}
