\subsection{2020-2021年度高等代数A(下)冬季学期试卷}

\begin{center}
    Texer:萌小小\quad Reviewer:C++ \\
\end{center}

\subsubsection*{填空题(每空2分,共20分)}

\begin{enumerate}
    \item 代数学基本定理是\underline{每一个次数 $\geq 1$ 的复系数多项式在 $\mathbb{C}$ 中都有根}.
    \item 设 $A$ 是 2021 维欧式空间 $V$ 的一组基的度量矩阵,则 $A$ 的符号差为 $\underline{2021}, A^*$ 秩为\underline{2021}.
    \item 复数域上不可约多项式的次数为\underline{1} ,实数域上不可约多项式的次数为\underline{$1,2$}.
    \item 如果不可约多项式 $p(x)$ 是 $f(x)$ 的 $k$ 重因式 $(k \geq 1)$ ,那么它是微商 $f^\prime(x)$ 的\underline{$k-1$ }重因式.
    \item 矩阵 $A$ 为欧式空间中由标准正交基到标准正交基的过渡矩阵,则 $|A|=\underline{\pm 1}$.
    \item 全体 2021 阶复对称矩阵按合同共分为\underline{2022}类.
    \item 两个有限维欧式空间同构的充分必要条件是它们的\underline{维数相同}.
    \item 同一线性变换在不同基下的矩阵的迹必\underline{相等}.
\end{enumerate}

\subsubsection*{判断题(每题2分,共10分,正确的填T,错的填F)}

\begin{enumerate}
    \item 属于不同特征根的特征向量线性无关.\hfill (T)
    \item 实对称矩阵$A$为负定的充分必要条件是$A$的所有顺序主子式都小于0. \hfill (F)
    \item 任何有限维欧式空间都存在标准正交基.\hfill (T)
    \item 实对称矩阵的特征值可以不是实数. \hfill (F)
    \item 奇数次实系数多项式不一定有实根.\hfill (F)
\end{enumerate}

\subsubsection*{计算题(共30分)}

\begin{enumerate}
    \item (10分)设三维欧式空间 $V$ 在基 $\alpha_1, ~ \alpha_2, ~ \alpha_3$ 下的度量矩阵是
    \begin{equation*}
    T=\left(\begin{array}{lll}
    1 & 1 & 1 \\
    1 & 2 & 3 \\
    1 & 3 & 6
    \end{array}\right)
    \end{equation*}
    求 $V$ 的一组标准正交基(用 $\alpha_1, \alpha_2, \alpha_3$ 的线性组合).
    \begin{solution}
        对 $(T, I)$ 作系列合同变换,将 $T$ 化为单位矩阵 $I$ ,
        \begin{equation*}
        \left[\begin{array}{rrrrrr}
        1 & 1 & 1 & 1 & 0 & 0 \\
        1 & 2 & 3 & 0 & 1 & 0 \\
        1 & 3 & 6 & 0 & 0 & 1
        \end{array}\right] \rightarrow \cdots \rightarrow\left[\begin{array}{cccccc}
        1 & 0 & 0 & 1 & 0 & 0 \\
        0 & 1 & 0 & -1 & 1 & 0 \\
        0 & 0 & 1 & 1 & -2 & 1
        \end{array}\right]
        \end{equation*}
        可以得到
        \begin{equation*}
        P=\left[\begin{array}{ccc}
        1 & 0 & 0 \\
        -1 & 1 & 0 \\
        1 & -2 & 1
        \end{array}\right]^{\prime}=\left[\begin{array}{ccc}
        1 & -1 & 1 \\
        0 & 1 & -2 \\
        0 & 0 & 1
        \end{array}\right]
        \end{equation*}
        则 $P^{\prime} T P=I$ .取 $\left(\beta_1, \beta_2, \beta_3\right)=\left(\alpha_1, \alpha_2, \alpha_3\right) P$ ,即
        \begin{equation*}
            \beta_1=\alpha_1,\beta_2=-\alpha_1+\alpha_2,\beta_3=\alpha_1-2\alpha_2+\alpha_3.
        \end{equation*}
        则内积在基$\beta_1,\beta_2,\beta_3$下的度量矩阵为$P^\prime TP=1$,所以$\beta_1,\beta_2,\beta_3$即为所求标准正交基.
    \end{solution}
    \item (10分)求 $f(x)=x^4-4 x^3+1$ 与 $g(x)=x^3-3 x^2+1$ 的首项系数为 1 的最大公因式.
    \begin{solution}
      1.
    \end{solution}
    \item (10分) 求用正交线性变换化二次型为标准型$f(x_1,x_2,x_3)=x_1x_2+x_2x_3+x_3x_1$.
\end{enumerate}

\subsubsection*{证明题(共40分)}

\begin{enumerate}
    \item (10分)证明实数域上二次型规范形的唯一性.
    \item (10分)设 $\sigma, \tau \in L\left(\mathbb{C}^n\right)$ 且 $\sigma+\tau+\sigma \tau=0$ ,求证:
    \begin{enumerate}
        \item[(1)] 若$\lambda$是 $\sigma$ 的一个特征值,则 $V_\lambda$ 是$\tau$的不变子空间.
        \item[(2)] $\sigma, \tau$ 至少有一个公共的特征向量.
    \end{enumerate}
    \item (10分)设 $A$ 为 $n$ 维欧氏空间中一组基的度量矩阵,$B$为半正定矩阵.求证$AB$的特征值全部是非负实数.
    \item (10分)
\end{enumerate}






