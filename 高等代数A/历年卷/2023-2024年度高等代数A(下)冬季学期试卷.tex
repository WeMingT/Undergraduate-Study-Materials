\subsection{2023-2024年度高等代数A(下)冬季学期试卷}

\subsubsection*{填空题}

\begin{enumerate}
    \item 多项式 \(f(x)\) 除以 \(ax-b(a\neq 0)\) 所得的余式是\underline{\hspace{2cm}}.
    \item \(2x^{4}-x^{3}+2x-3=0\) 的所有有理根为\underline{\hspace{2cm}}.
    \item 二次型 \(f(x_{1},x_{2},x_{3})=(x_{1},x_{2},x_{3})\begin{pmatrix}1 & 2 & 3 \\ 4 & 5 & 6 \\ 7 & 8 & 9\end{pmatrix}\begin{pmatrix}x_{1} \\ x_{2} \\ x_{3}\end{pmatrix}\) 的矩阵是\underline{\hspace{2cm}}.
    \item 矩阵 \begin{equation*}\begin{pmatrix}0 & 0 & a_{3} \\ 1 & 0 & -a_{2} \\ 0 & 1 & -a_{1}\end{pmatrix}\end{equation*} 的不变因子是\underline{\hspace{2cm}}.
    \item 设 \(V\) 是三维欧氏空间,\(\varepsilon_{1},\varepsilon_{2},\varepsilon_{3}\) 是 \(V\) 的标准正交基,如果 \(\alpha_{1}=\varepsilon_{1}\),\(\alpha_{2}=\varepsilon_{1}+\varepsilon_{2}\),\(\alpha_{3}=\varepsilon_{1}+\varepsilon_{2}+\varepsilon_{3}\),则基 \(\alpha_{1},\alpha_{2},\alpha_{3}\) 的度量矩阵为 \begin{equation*}\begin{pmatrix}1 & 1 & 1 \\ 1 & 2 & 2 \\ 1 & 2 & 3\end{pmatrix}\end{equation*}.
\end{enumerate}

\subsubsection*{选择题}

\begin{enumerate}
    \item 数域 \(P\) 上多项式 \(f(x)\) 在 \(P\) 上无重因式的条件是 \((f(x),f'(x))=1\).
    \begin{enumerate}
        \item[(A)] 充要
        \item[(B)] 充分非必要
        \item[(C)] 必要非充分
        \item[(D)] 非充分非必要
    \end{enumerate}

    \item 二次型 \(f(x_{1},x_{2},x_{3})=x_{1}^{2}+x_{2}^{2}+x_{1}x_{3}+x_{2}x_{3}\) 是 ( ) 二次型。
    \begin{enumerate}
        \item[(A)] 正定
        \item[(B)] 不定
        \item[(C)] 负定
        \item[(D)] 半正定
    \end{enumerate}

    \item \(A=\begin{pmatrix}1 & 2 \\ 2 & 1\end{pmatrix}\),在实数域 \(\mathbb{R}\) 上与 \(A\) 合同的矩阵为 ( )。
    \begin{enumerate}
        \item[(A)] \(\begin{pmatrix}-2 & 1 \\ 1 & -2\end{pmatrix}\)
        \item[(B)] \(\begin{pmatrix}2 & -1 \\ -1 & 2\end{pmatrix}\)
        \item[(C)] \(\begin{pmatrix}2 & 1 \\ 1 & 2\end{pmatrix}\)
        \item[(D)] \(\begin{pmatrix}1 & -2 \\ -2 & 1\end{pmatrix}\)
    \end{enumerate}

    \item ( ) 不是 \(\lambda\)-矩阵的初等变换。
    \begin{enumerate}
        \item[(A)] 矩阵的某一行(列)乘非零常数 \(c\)
        \item[(B)] 矩阵的某一行(列)乘非零常数 \(\frac{1}{c}\)
        \item[(C)] 矩阵的某一行(列)加另一行(列)的 \(\lambda\) 倍
        \item[(D)] 矩阵的某一行(列)加另一行(列)的 \(\frac{1}{\lambda}\) 倍
    \end{enumerate}

    \item 在 \(\mathbb{R}^{4}\) 中, \(\alpha=(2,1,3,2)\),\(\beta=(1,2,x,1)\),\((\alpha,\beta)=0\),则 \(x\) = ( )。
    \begin{enumerate}
        \item[(A)] -1
        \item[(B)] 1
        \item[(C)] -2
        \item[(D)] 2
    \end{enumerate}
\end{enumerate}

\subsubsection*{计算题}

\begin{enumerate}
    \item (10分) 用非退化线性替换化二次型
    \begin{equation*}
    f(x_{1},x_{2},x_{3})=x_{1}^{2}+x_{2}^{2}+x_{3}^{2}+2x_{1}x_{2}+2x_{1}x_{3}+4x_{2}x_{3}
    \end{equation*}
    为规范形,并求相应的线性替换和符号差。

    \item (10分) 设 \(T\) 是线性空间 \(V\) 上的线性变换,\(\alpha\) 是 \(V\) 的非零向量,若向量组 \(\alpha,T\alpha,\dots,T^{m-1}\alpha\) 线性无关。
    \begin{enumerate}
        \item[(1)] 证明:子空间 \(W=L(\alpha,T\alpha,\dots,T^{m-1}\alpha)\) 是 \(T\) 的不变子空间。
        \item[(2)] 求 \(T\) 在子空间 \(W\) 的基 \(\alpha,T\alpha,\dots,T^{m-1}\alpha\) 下的矩阵。
    \end{enumerate}

    \item (10分) 求
    \begin{equation*}
    \begin{pmatrix}-1 & 2 & 6 \\ 1 & 7 & 25 \\ 0 & -2 & -7\end{pmatrix}
    \end{equation*}
    的 Jordan 标准型。

    \item (10分) .....,求 \(T^{-1}e^{i\pi A/2}T\)。
\end{enumerate}

\subsubsection*{证明题}

\begin{enumerate}
    \item (10分) 设 \(f(x),g(x)\in P[x]\),\(a,b,c,d \in P\),\(f_{1}(x)=af(x)+bg(x)\),\(g_{1}(x)=cf(x)+dg(x)\),\(ad \neq bc\),\((f(x),g(x))=1\),证明:\((f_{1}(x),f_{1}(x)+g_{1}(x))=1\)。

    \item (10分) 设 \(\alpha_{1},\alpha_{2},\dots,\alpha_{m}\) 和 \(\beta_{1},\beta_{2},\dots,\beta_{m}\) 是 \(n\) 维欧式空间 \(V\) 中的两个向量组,证明存在一个正交变换 \(T\) 使得 \(T\alpha_{i}=\beta_{i}(1 \leq i \leq m)\) 的充要条件是 \((\alpha_{i},\alpha_{j})=(\beta_{i},\beta_{j})(1 \leq i,j \leq m)\)。
\end{enumerate}