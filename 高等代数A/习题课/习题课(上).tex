\subsection{习题课(上)}

\subsubsection{行列式}

\begin{problem}
    计算下述 $n$ 阶行列式:
\[
\left|\begin{array}{ccccc}
0 & a_1 & 0 & \cdots & 0 \\
0 & 0 & a_2 & \cdots & 0 \\
\vdots & \vdots & \vdots & & \vdots \\
0 & 0 & 0 & \cdots & a_{n-1} \\
a_n & 0 & 0 & \cdots & 0
\end{array}\right| .
\]
\end{problem}

\begin{solution}
    此行列式的每一行有 $n-1$ 个元素为 0 , 因此在它的完全展开式中, 可能不为 0的项只有一项,从而这个行列式的值为
\[
(-1)^{r(23 \cdots n 1)} a_1 a_2 \cdots a_{n-1} a_n=(-1)^{n-1} a_1 a_2 \cdots a_{n-1} a_n .
\]
\end{solution}

\begin{problem}
    计算下述 $n$ 阶行列式:
\[
\left|\begin{array}{ccccc}
0 & 0 & \cdots & 0 & a_1 \\
0 & 0 & \cdots & a_2 & 0 \\
\vdots & \vdots & & \vdots & \vdots \\
0 & a_{n-1} & \cdots & 0 & 0 \\
a_n & 0 & \cdots & 0 & 0
\end{array}\right| .
\]
\end{problem}

\begin{solution}
    \[
\begin{aligned}
\text { 原式 } & =(-1)^{r(n(n-1) \cdots 21)} a_1 a_2 \cdots a_{n-1} a_n \\
& =(-1)^{\frac{n(n-1)}{2}} a_1 a_2 \cdots a_{n-1} a_n .
\end{aligned}
\]
\end{solution}

\begin{problem}
    计算 $n$ 阶行列式 $(n \geqslant 2)$ :
\[
\left|\begin{array}{ccccc}
x_1-a_1 & x_2 & x_3 & \cdots & x_n \\
x_1 & x_2-a_2 & x_3 & \cdots & x_n \\
x_1 & x_2 & x_3-a_3 & \cdots & x_n \\
\vdots & \vdots & \vdots & & \vdots \\
x_1 & x_2 & x_3 & \cdots & x_n-a_n
\end{array}\right|,
\]
其中 $a_i \neq 0, i=1,2, \cdots, n$.
\end{problem}

\begin{solution}
    先把第 1 行的 $(-1)$ 倍分别加到第 $2,3, \cdots, n$ 行上, 然后各列分别提出公因子 $a_1, a_2, \cdots, a_n$ ;
\[
\begin{aligned}
\text { 原式 } & =\left|\begin{array}{ccccc}
x_1-a_1 & x_2 & x_3 & \cdots & x_n \\
a_1 & -a_2 & 0 & \cdots & 0 \\
a_1 & 0 & -a_3 & \cdots & 0 \\
\vdots & \vdots & \vdots & & \vdots \\
a_1 & 0 & 0 & \cdots & -a_n
\end{array}\right| \\
& =a_1 a_2 a_3 \cdots a_n\left|\begin{array}{ccccc}
\frac{x_1}{a_1}-1 & \frac{x_2}{a_2} & \frac{x_3}{a_3} & \cdots & \frac{x_n}{a_n} \\
1 & -1 & 0 & \cdots & 0 \\
1 & 0 & -1 & \cdots & 0 \\
\vdots & \vdots & \vdots & & \vdots \\
1 & 0 & 0 & \cdots & -1
\end{array}\right| \\
& =a_1 a_2 a_3 \cdots a_n\left|\begin{array}{ccccc}
\sum_{i=1}^n \frac{x_i}{a_i}-1 & \frac{x_2}{a_2} & \frac{x_3}{a_3} & \cdots & \frac{x_n}{a_n} \\
0 & -1 & 0 & \cdots & 0 \\
0 & 0 & -1 & \cdots & 0 \\
\vdots & \vdots & \vdots & & \vdots \\
0 & 0 & 0 & \cdots & -1
\end{array}\right| \\
& =(-1)^{n-1} a_1 a_2 a_3 \cdots a_n\left(\sum_{i=1}^n \frac{x_i}{a_i}-1\right) .
\end{aligned}
\]
\end{solution}

\begin{problem}
    计算实数域上 $n$ 阶三对角线行列式:
\[
D_n=\left|\begin{array}{cccccccc}
a & b & 0 & 0 & \cdots & 0 & 0 & 0 \\
c & a & b & 0 & \cdots & 0 & 0 & 0 \\
0 & c & a & b & \cdots & 0 & 0 & 0 \\
\vdots & \vdots & \vdots & \vdots & & \vdots & \vdots & \vdots \\
0 & 0 & 0 & 0 & \cdots & c & a & b \\
0 & 0 & 0 & 0 & \cdots & 0 & c & a
\end{array}\right| .
\]
\end{problem}

\begin{solution}
    \[
D_n= \begin{cases}\displaystyle \frac{\alpha_1^{n+1}-\beta_1^{n+1}}{\alpha_1-\beta_1}, & \text { 当 } a^2 \neq 4 b c, \\ \displaystyle (n+1) \frac{a^n}{2^n}, & \text { 当 } a^2=4 b c,\end{cases}
\]
其中 $\alpha_1, \beta_1$ 是方程 $x^2-a x+b c=0$ 的两个根.
\end{solution}

\begin{problem}
    求$\begin{vmatrix}
        \lambda &0 &0 &\dots &0 &a_n\\
        -1 &\lambda &0 &\dots &0 &a_{n-1}\\
        0 &-1 &\lambda &\dots &0 &a_{n-2}\\
        \vdots &\vdots &\vdots & &\vdots &\vdots\\
        0 &0 &0 &\dots &\lambda &a_2\\
        0 &0 &0 &\dots &-1 &\lambda +a_1
    \end{vmatrix}.$
\end{problem}

\begin{solution}
    
\end{solution}

\begin{problem}
    求$n$阶行列式$\begin{vmatrix}
        x &a &a &\dots &a\\
        a &x &a &\dots &a\\
        \vdots &\vdots &\vdots &  &\vdots\\
        a &a &a &\dots &x
    \end{vmatrix}.$
\end{problem}

\begin{solution}
    
\end{solution}

\begin{problem}
    设 $\boldsymbol{A}, \boldsymbol{B}$ 是 $n$ 阶矩阵, 求证:
$$
\left|\begin{array}{ll}
\boldsymbol{A} & \boldsymbol{B} \\
\boldsymbol{B} & \boldsymbol{A}
\end{array}\right|=|\boldsymbol{A}+\boldsymbol{B}||\boldsymbol{A}-\boldsymbol{B}|
$$
\end{problem}

\begin{proof}
    将分块矩阵的第二行加到第一行上, 再将第二列减去第一列, 可得
$$
\left(\begin{array}{cc}
A & B \\
B & A
\end{array}\right) \rightarrow\left(\begin{array}{cc}
A+B & A+B \\
B & A
\end{array}\right) \rightarrow\left(\begin{array}{cc}
A+B & O \\
B & A-B
\end{array}\right)
$$
第三类分块初等变换不改变行列式的值, 因此可得
$$
\left|\begin{array}{ll}
A & B \\
B & A
\end{array}\right|=\left|\begin{array}{cc}
A+B & O \\
B & A-B
\end{array}\right|=|A+B \| A-B|
$$
\end{proof}

\begin{problem}
    设$A,B,C,D \in \mathbb{K}^{n \times n}$且$A$可逆,若$AC=CA$,证明:
    $\begin{vmatrix}
        A &B\\C &D
    \end{vmatrix}=|AD-CB|.$
\end{problem}

\begin{proof}
    
\end{proof}

\subsubsection{矩阵}

\begin{problem}
    计算 $A^m$, 其中 $m$ 是正整数, 且
\[
A=\left(\begin{array}{ll}
2 & 3 \\
0 & 2
\end{array}\right)
\]
\end{problem}

\begin{solution}
    \[
A=\left(\begin{array}{ll}
2 & 3 \\
0 & 2
\end{array}\right)=\left(\begin{array}{ll}
2 & 0 \\
0 & 2
\end{array}\right)+\left(\begin{array}{ll}
0 & 3 \\
0 & 0
\end{array}\right)=2 I+3 B,
\]
其中 $B=\left(\begin{array}{ll}0 & 1 \\ 0 & 0\end{array}\right)$.
直接计算得,
\[
B^2=\left(\begin{array}{ll}
0 & 1 \\
0 & 0
\end{array}\right)\left(\begin{array}{ll}
0 & 1 \\
0 & 0
\end{array}\right)=\left(\begin{array}{ll}
0 & 0 \\
0 & 0
\end{array}\right) .
\]
由于 $(2 I)(3 B)=(3 B)(2 I)$, 因此由二项式定理得
\[
\begin{aligned}
A^m & =(2 I+3 B)^m=(2 I)^m+C_m^1(2 I)^{m-1}(3 B)=2^m I+2^{m-1} \cdot 3 m B \\
& =\left(\begin{array}{cc}
2^m & 2^{m-1} \cdot 3 m \\
0 & 2^m
\end{array}\right)
\end{aligned}
\]
\end{solution}

\begin{problem}
    计算$A^k$,其中$A=\begin{pmatrix}
        \cos \theta & -\sin \theta \\
        \sin \theta & \cos \theta
    \end{pmatrix}$.
\end{problem}

\begin{solution}
    归纳证明

$$A^k=\begin{pmatrix}\cos k\theta&-\sin k\theta\\\sin k\theta&\cos k\theta\end{pmatrix}.$$

当$k=1$时,根据条件即得. 假设当$k$时结论成立,则当$k+1$时,

$$\begin{aligned}A^{k+1}=A^k\cdot A&=\begin{pmatrix}\cos k\theta&-\sin k\theta\\\sin k\theta&\cos k\theta\end{pmatrix}\begin{pmatrix}\cos\theta&-\sin\theta\\\sin\theta&\cos\theta\end{pmatrix}\\&=\begin{pmatrix}\cos(k+1)\theta&-\sin(k+1)\theta\\\sin(k+1)\theta&\cos(k+1)\theta\end{pmatrix}.\end{aligned}$$

这就证明了结论.
\end{solution}

\begin{problem}
    求下述 $n$ 级矩阵 $A$ 的逆矩阵 $(n \geqslant 2)$ :
\[
A=\left(\begin{array}{ccccc}
0 & 1 & 1 & \cdots & 1 \\
1 & 0 & 1 & \cdots & 1 \\
1 & 1 & 0 & \cdots & 1 \\
\vdots & \vdots & \vdots & & \vdots \\
1 & 1 & 1 & \cdots & 0
\end{array}\right)
\]
\end{problem}

\begin{solution}{(方法一)}
    设 $\boldsymbol{\alpha}=(1,1, \cdots, 1)^{\prime}$, 则 $\boldsymbol{A}=-\boldsymbol{I}_n+\boldsymbol{\alpha} \boldsymbol{\alpha}^{\prime}$. 设 $\boldsymbol{B}=c \boldsymbol{I}_n+d \boldsymbol{\alpha} \boldsymbol{\alpha}^{\prime}$, 则通过简单的计算可知 $\boldsymbol{A B}=-c \boldsymbol{I}_n+(c+(n-1) d) \boldsymbol{\alpha} \boldsymbol{\alpha}^{\prime}$. 令 $c=-1, c+(n-1) d=0$, 则 $d=\frac{1}{n-1}$, 于是 $\boldsymbol{A B}=\boldsymbol{I}_n$, 从而 $\boldsymbol{A}^{-1}=\boldsymbol{B}=-\boldsymbol{I}_n+\frac{1}{n-1} \boldsymbol{\alpha} \boldsymbol{\alpha}^{\prime}$.
\end{solution}

\begin{solution}{(方法二)}
    取 $n \times 2 n$ 矩阵
\[
B=\left(\begin{array}{ccccc|ccccc}
0 & 1 & 1 & \cdots & 1 & 1 & 0 & 0 & \cdots & 0 \\
1 & 0 & 1 & \cdots & 1 & 0 & 1 & 0 & \cdots & 0 \\
1 & 1 & 0 & \cdots & 1 & 0 & 0 & 1 & \cdots & 0 \\
\vdots & \vdots & \vdots & \ddots & \vdots & \vdots & \vdots & \vdots & \ddots & \vdots \\
1 & 1 & 1 & \cdots & 0 & 0 & 0 & 0 & \cdots & 1
\end{array}\right) .
\]
把矩阵 $B$ 的第 $2,3, \ldots, n$ 行都加到第 1 行,矩阵 $B$ 化为
\[
B_1=\left(\begin{array}{ccccc:ccccc}
n-1 & n-1 & n-1 & \cdots & n-1 & 1 & 1 & 1 & \cdots & 1 \\
1 & 0 & 1 & \cdots & 1 & 0 & 1 & 0 & \cdots & 0 \\
1 & 1 & 0 & \cdots & 1 & 0 & 0 & 1 & \cdots & 0 \\
\vdots & \vdots & \vdots & \ddots & \vdots & \vdots & \vdots & \vdots & \ddots & \vdots \\
1 & 1 & 1 & \cdots & 0 & 0 & 0 & 0 & \cdots & 1
\end{array}\right) .
\]
用 $\frac{-1}{n-1}$ 遍乘矩阵 $B_1$ 的第 1 行,然后分别加到第 $2,3, \ldots, n$ 行,矩阵 $B_1$ 化为
\[
B_2=\left(\begin{array}{ccccc|ccccc}
1 & 1 & 1 & \cdots & 1 & \frac{1}{n-1} & \frac{1}{n-1} & \frac{1}{n-1} & \cdots & \frac{1}{n-1} \\
0 & -1 & 0 & \cdots & 0 & -\frac{1}{n-1} & \frac{n-2}{n-1} & -\frac{1}{n-1} & \cdots & -\frac{1}{n-1} \\
0 & 0 & -1 & \cdots & 0 & -\frac{1}{n-1} & -\frac{1}{n-1} & \frac{n-2}{n-1} & \cdots & -\frac{1}{n-1} \\
\vdots & \vdots & \vdots & \ddots & \vdots & \vdots & \vdots & \vdots & \ddots & \vdots \\
0 & 0 & 0 & \cdots & -1 & -\frac{1}{n-1} & -\frac{1}{n-1} & -\frac{1}{n-1} & \cdots & \frac{n-2}{n-1}
\end{array}\right) .
\]
矩阵 $B_2$ 的第 $2,3, \ldots, n$ 行都加到第 1 行,然后用 -1 分别乘以第 $2,3, \ldots, n$ 行,矩阵$B_2$ 变为
\[
B_2=\left(\begin{array}{ccccc:ccccc}
1 & 0 & 0 & \cdots & 0 & -\frac{n-2}{n-1} & \frac{1}{n-1} & \frac{1}{n-1} & \cdots & \frac{1}{n-1} \\
0 & 1 & 0 & \cdots & 0 & \frac{1}{n-1} & -\frac{n-2}{n-1} & \frac{1}{n-1} & \cdots & \frac{1}{n-1} \\
0 & 0 & 1 & \cdots & 0 & \frac{1}{n-1} & \frac{1}{n-1} & -\frac{n-2}{n-1} & \cdots & \frac{1}{n-1} \\
\vdots & \vdots & \vdots & \ddots & \vdots & \vdots & \vdots & \vdots & \ddots & \vdots \\
0 & 0 & 0 & \cdots & 1 & \frac{1}{n-1} & \frac{1}{n-1} & \frac{1}{n-1} & \cdots & -\frac{n-2}{n-1}
\end{array}\right) .
\]
由此求得, $A^{-1}=\left(c_{i j}\right)$, 其中
\[
\begin{cases}c_{i j}=\frac{1}{n-1}, & \text { 当 } 1 \leqslant i \neq j \leqslant n \text { 时; } \\ c_{i i}=-\frac{n-2}{n-1}, & \text { 当 } i=1,2, \ldots, n \text { 时. }\end{cases}
\]
\end{solution}

\begin{solution}{(方法三)}
    设 $J$ 为基础循环矩阵,则 $\boldsymbol{A}=\boldsymbol{J}+\boldsymbol{J}^2+\cdots+J^{n-1}$ . 设 $B=c \boldsymbol{I}_n+\boldsymbol{J}+$ $J^2+\cdots+J^{n-1}$, 其中 $c$ 为待定系数, 则通过简单的计算可得
\[
\boldsymbol{A B}=(n-1) \boldsymbol{I}_n+(c+n-2)\left(\boldsymbol{J}+\boldsymbol{J}^2+\cdots+\boldsymbol{J}^{n-1}\right) .
\]
只要令 $c=2-n$, 则 $\boldsymbol{A B}=(n-1) \boldsymbol{I}_n$, 于是 $\boldsymbol{A}^{-1}=\frac{1}{n-1} \boldsymbol{B}$.
\end{solution}

\begin{solution}{(方法四)}
    设 $\boldsymbol{\alpha}=(1,1, \cdots, 1)^{\prime}$ ,则 $\boldsymbol{A}=-\boldsymbol{I}_n+\boldsymbol{\alpha} \boldsymbol{\alpha}^{\prime}$ 。由 Sherman-Morrison 公式可得
\[
\begin{aligned}
A^{-1}=\left(-I_n+\alpha \alpha^{\prime}\right)^{-1} & =\left(-I_n\right)^{-1}-\frac{1}{1+\boldsymbol{\alpha}^{\prime}\left(-I_n\right)^{-1} \boldsymbol{\alpha}}\left(-I_n\right)^{-1} \alpha \boldsymbol{\alpha}^{\prime}\left(-I_n\right)^{-1} \\
& =-I_n+\frac{1}{n-1} \boldsymbol{\alpha} \boldsymbol{\alpha}^{\prime} .
\end{aligned}
\]
\end{solution}



\begin{problem}
    设
\[
B=\left(\begin{array}{cc}
0 & B_1 \\
B_2 & 0
\end{array}\right),
\]
其中 $B_1, B_2$ 分别是 $r$ 级、 $s$ 级矩阵。求 $B$ 可逆的充分必要条件; 当 $B$ 可逆时, 求 $B^{-1}$.
\end{problem}

\begin{solution}
    $|B|=(-1)^r\left|B_1\right|\left|B_2\right|$ .于是
$B$ 可逆 $\Longleftrightarrow|B| \neq 0 \Longleftrightarrow\left|B_1\right| \neq 0$ 且 $\left|B_2\right| \neq 0 \Longleftrightarrow B_1, B_2$ 都可逆.
当 $B$ 可逆时,由于
\[
\left(\begin{array}{cc}
0 & B_1 \\
B_2 & 0
\end{array}\right)\left(\begin{array}{ll}
0 & B_2^{-1} \\
B_1^{-1} & 0
\end{array}\right)=\left(\begin{array}{cc}
I_r & 0 \\
0 & I_s
\end{array}\right),
\]
因此
\[
B^{-1}=\left(\begin{array}{cc}
0 & B_2^{-1} \\
B_1^{-1} & 0
\end{array}\right)
\]
\end{solution}

\begin{problem}
    (Frobenius 秩不等式) 证明: $\operatorname{rk}(R S T) \geq \operatorname{rk}(R S)+\operatorname{rk}(S T)-\operatorname{rk}(S)$, 前提是这些线性映射的合成有意义.
\end{problem}

\begin{remark}
    原题为矩阵版本.
\end{remark}

\begin{proof}
    \[
\begin{aligned}
& \text {应用 } \\
& \qquad \begin{aligned}
\operatorname{dim}(\operatorname{im}(R S T)) & =\operatorname{dim}(\operatorname{im}(S T))-\operatorname{dim}(\operatorname{im}(S T) \cap \operatorname{ker}(R)) \\
& \geq \operatorname{dim}(\operatorname{im}(S T))-\operatorname{dim}(\operatorname{im}(S) \cap \operatorname{ker}(R)), \\
\operatorname{dim}(\operatorname{im}(R S)) & =\operatorname{dim}(\operatorname{im}(S))-\operatorname{dim}(\operatorname{im}(S) \cap \operatorname{ker}(R)) .
\end{aligned}
\end{aligned}
\]
\end{proof}

\begin{problem}
    证明: (1) 若$A$为可逆对称矩阵,则$A^{-1}=(A^{-1})^{T}$,若$|A|\neq0,A^{T}=-A$,则$A^{-1}=-(A^{-1})^{T}$.\\
    (2) 不存在奇数级可逆反对称矩阵.
\end{problem}

\begin{proof}
    
\end{proof}

\begin{problem}
    证明: (1) 两个上(下)三角矩阵乘积仍是上(下)三角阵.\\
    (2) 可逆上(下)三角矩阵的逆仍是上(下)三角矩阵.
\end{problem}

\begin{proof}
    
\end{proof}

\begin{problem}
    设有$n$阶矩阵$A$,证明:存在可逆矩阵$P$使得$PAP^{-1}$的后$n-\operatorname{r}(A)$行全为零.
\end{problem}

\begin{proof}
    
\end{proof}

\begin{problem}
    求证: 任一 $n$ 阶方阵均可表示为一个对称阵与一个反对称阵之和.
\end{problem}

\begin{proof}
设 $\boldsymbol{A}$ 是 $n$ 阶方阵, 则 $\boldsymbol{A}+\boldsymbol{A}^{\prime}$ 是对称阵, $\boldsymbol{A}-\boldsymbol{A}^{\prime}$ 是反对称阵, 并且
$$
\boldsymbol{A}=\frac{1}{2}\left(\boldsymbol{A}+\boldsymbol{A}^{\prime}\right)+\frac{1}{2}\left(\boldsymbol{A}-\boldsymbol{A}^{\prime}\right)
$$
\end{proof}

\begin{remark}
    上例中的 $\frac{1}{2}\left(\boldsymbol{A}+\boldsymbol{A}^{\prime}\right)$ 称为 $\boldsymbol{A}$ 的对称化, $\frac{1}{2}\left(\boldsymbol{A}-\boldsymbol{A}^{\prime}\right)$ 称为 $\boldsymbol{A}$ 的反对称化. 上述分解使得我们可以利用对称阵和反对称阵的众多性质去研究方阵的性质.
\end{remark}

\begin{problem}
    设$A$为$n$阶实矩阵,证明:$\exists t \in \mathbb{R}$使得$A+tI_n$可逆.
\end{problem}

\begin{proof}
    
\end{proof}

\begin{problem}
    设$A,B$为实方阵,证明:$A,B$在$\mathbb{R}$上相似$\Leftrightarrow$$A,B$在$\mathbb{C}$上相似.
\end{problem}

\begin{proof}
    
\end{proof}



\begin{problem}
    
\end{problem}

\begin{solution}
    
\end{solution}

\begin{problem}
    
\end{problem}

\begin{proof}
    
\end{proof}