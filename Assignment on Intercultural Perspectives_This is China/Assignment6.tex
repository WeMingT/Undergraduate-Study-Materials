\section{Assignment-6}

\begin{question}{}{}
    Design a case to introduce Chinese festivals to your foreign friend who just arrives in Shanghai. (300-400words)
\end{question}

Welcome to \textbf{Shanghai}, \textbf{China}, and thank you for your interest in \textbf{Chinese festivals}. Now, let me introduce them to you. I hope this will be helpful.

Did you enjoy playing with water when you were a child? Are you familiar with a festival where water play is the main activity? As you may know, \textbf{China} is a country where many ethnic groups coexist harmoniously. Although I am \textbf{Han Chinese}, many ethnic groups also have varieties of festivals, among which are the \textbf{Tibetan New Year} of Tibetan people and the \textbf{Water Splashing Festival} of the Dai people. Taking the \textbf{Water Splashing Festival} of the Dai people as an example, this festival marks the Dai New Year and usually takes place in mid-April, coinciding with the hottest time of the year. During the festival, people splash water on each other as a form of cleansing and blessing. It is believed that the water washes away bad luck and purifies individuals for the coming year. Participants often use buckets, water guns, or simply their hands to splash water, and the streets become filled with joy and laughter as everyone gets involved.

Unlike the calendar system adopted in the United States, ancient China adopted a \textbf{lunisolar calendar}, and some festivals have the same numbers for the day and the month. For example, the \textbf{Spring Festival} on the first day of the first month, the \textbf{Dragon Boat Festival} on the fifth day of the fifth month, and the \textbf{Double Ninth Festival} on the ninth day of the ninth month. However, they are not the products of some number games. In the eyes of the ancient Chinese, these days were considered inauspicious, therefore, people would take a bath, drive off bad luck, offer sacrifices to gods, and pray for good luck. For instance, on the \textbf{Spring Festival}, people would clean the house, set off firecrackers and fireworks, decorate the house, stay up, and feast. On the \textbf{Dragon Boat Festival}, people would hang Chinese mugwort and calamus on doors, wear perfume pouches, and drink realgar wine.

In recent years, some new festivals have emerged in China, such as November 11, also known as the \textbf{Double 11 Festival}, which is similar to \textbf{Black Friday} in America. Additionally, the \textbf{China Shanghai International Arts Festival} is currently underway, and if you are interested, I welcome you to join me in exploring it.
