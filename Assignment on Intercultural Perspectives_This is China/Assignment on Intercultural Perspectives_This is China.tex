\documentclass[12pt,lang=en,thmcnt=section]{elegantbook}
\usepackage{ctex}
\usepackage{longtable}
\usepackage{tabularx}
\usepackage{ltablex}

\elegantnewtheorem{question}{Assignment}{thmstyle}{exfancy}
\elegantnewtheorem{part}{Part}{thmstyle}{exfancy}

\title{Assignment}
\subtitle{Intercultural Perspectives: This is China}

\author{萌小小}
\bioinfo{学号}{23******}
\institute{Shanghai University}
\date{\today}

\cover{cover.jpg}
\logo{shulogo.png}

\begin{document}
\maketitle
\frontmatter

\tableofcontents

\mainmatter

\chapter{Assignment}

\section{Assignment-1}

\begin{question}{}{}
    To make some comments on the challenges for China's intercultural communication in English(200-300 words).
\end{question}

In China's intercultural communication, there are primarily three challenges:

\textbf{Challenge 1}: \textbf{Language Barrier}

Aside from a few countries like Japan, which share a similar writing system with China, the languages of most countries greatly differ from Chinese. The high cost of learning and the difficulty of disseminating the language necessitate ``translation" as a means of intercultural communication, requiring additional human and financial resources. However, in countries where English is the official language, such a ``language barrier" does not exist.

Taking the classic work of Chinese traditional culture, the ``Tao Te Ching," as an example, its original Chinese text is written in classical Chinese, which is already challenging for modern Chinese to understand, let alone for foreign friends. The English translations of the ``Tao Te Ching" also face inevitable translation issues. For instance, the core term ``Tao" has been translated as ``way," ``dao," ``tao," among others, but it is undeniable that it is difficult to capture the profound essence of the Chinese character \textbf{道} through these translations.

\textbf{Challenge 2}: \textbf{Cultural Conflict}

When Chinese culture conflicts significantly with another culture in certain aspects, intercultural communication faces tremendous challenges. These challenges mainly stem from the resistance and skepticism of the inherent beliefs and aesthetic paradigms of that culture. For example, in some Middle Eastern countries where women wear veils and dress in black, colorful Chinese women's attire might encounter resistance and skepticism from religious customs. Similarly, while the mainstream aesthetic for men in the West is ``masculine," characterized by height and strength, in China, it is ``handsome," favoring slimness. Another example is the cultural differences between the East and the West in terms of family and filial piety depicted in the movie ``Hi, Mom.''

\textbf{Challenge 3}: \textbf{Political Status, Standpoint, and Policies}

Undeniably, in terms of political status, if China's political status is low, even with its splendid culture, it would still struggle with intercultural communication. After all, the burning of the Yuanming Yuan by the Eight-Nation Alliance was not intercultural communication but cultural destruction. Moreover, taking the Qing Dynasty as an example, its policy of resisting foreign cultures and implementing isolationism towards the end led to reciprocal resistance towards Chinese culture in other countries. This poses an insurmountable barrier for intercultural communication.

\section{Assignment-2}

\begin{question}{}{}
    Make comments on how Li Ziqi successfully tells the peach blossom story from the view point of Intercultural Communication. (300-400words)
\end{question}

There are various reasons why \textbf{Li Ziqi} has succeeded in storytelling, and four main reasons stand out.

From the perspective of the \textbf{intercultural speaker}, as the protagonist of the videos, \textbf{Li Ziqi} is young and beautiful, exuding no sense of threat but rather an amiable charm that evokes affection. Moreover, \textbf{Li Ziqi} is well-versed in Chinese traditional culture and its dissemination, choosing themes with thoughtful deliberation. The \textbf{peach blossom}'s universal appeal and status as a distinct symbol of Chinese culture, free from political, religious, or racial biases, make it an ideal cultural medium. For instance, it would not be suitable to use beef or mutton as cultural carriers for audiences including vegetarians or Muslims. Additionally, \textbf{Li Ziqi} integrates aspects of daily life—clothing, food, shelter, and travel—into her narrative, silently showcasing the essence of Chinese traditional culture in the mundanity of daily life, such as the creation of \textbf{Hanfu}, \textbf{peach blossom rice balls}, \textbf{peach blossom rice wine}, and \textbf{peach blossom cakes}, presenting these elements in an accessible manner, as if saying, ``It is right there."

From the perspective of the \textbf{audience}, while Westerners often enjoy camping, Chinese people prefer dining out at restaurants. Thus, \textbf{Li Ziqi}'s decision to conclude her video with a picnic in the peach grove alongside her grandmother, transitioning the scene from a confined space to an expansive orchard, and ending with an aerial shot, achieves multiple purposes at once. Furthermore, given the high degree of modernization in Europe and America, the natural ambiance \textbf{Li Ziqi} presents serves as a refreshing oasis. This resonates with the audience’s psychology and expectations, endowing the content with the potential for widespread appeal.

From the perspective of \textbf{strategies of communication}, \textbf{Li Ziqi} employs short videos for intercultural communication, crafted with exquisite quality, appealing to a broad audience. These videos focus on situational storytelling, with \textbf{Li Ziqi} actively involved in demonstrating processes, largely without dialogue or third-party narration, thus avoiding linguistic barriers. Even when text descriptions are used, they are accompanied by vivid imagery, making them self-explanatory. Similar to silent films, this approach provides a shared language for viewers, awakening common cultural elements among them. Moreover, this ``immersive" experience leaves a lasting impression on viewers.

From the perspective of the \textbf{content of the communication}, \textbf{Li Ziqi} not only promotes the tangible elements of Chinese traditional culture, such as \textbf{Hanfu} clothing, traditional dwellings, and cuisine like \textbf{peach blossom rice balls}, \textbf{peach blossom rice wine}, and \textbf{peach blossom cakes}, but also conveys the intangible spirit of this culture. She silently communicates the beautiful familial bond between herself and her grandmother—a sentiment that transcends borders and is a universal human emotion, reflecting China's tradition of respecting the elderly.

In addition, factors such as \textbf{Li Ziqi}'s massive following and the current wave of Chinese traditional culture spreading to the world also contribute to the success.

\section{Assignment-3}

\begin{question}{}{}
    Please finish the BBC video and build your vocabulary about cuizine when reviewing the 3 videos in this unit.
\end{question}

\begin{longtable}{|p{7cm}|p{7cm}|}
    \caption{Vocabulary About Cuisine} \label{tab:vocab_cuisine} \\
    \hline
    \textbf{English} & \textbf{Chinese} \\
    \hline
    \endfirsthead
    
    \caption[]{Vocabulary About Cuisine (continued)} \\
    \hline
    
    \endhead
    
    \hline
    \endfoot
    
    \hline
    \endlastfoot
    
    \multicolumn{2}{c}{\textbf{Influencing Factors}} \\ \hline
    geographical conditions & 地理条件 \\ \hline
    environment and climate diversity & 环境和气候多样性 \\ \hline
    people have wheaten food in the north of China and rice in the south & 南米北面 \\ \hline
    
    \multicolumn{2}{c}{\textbf{Four Flavors}} \\ \hline
    \textbf{Bashu} & \textbf{Qilu} \\ \hline
    Shredded Pork with Chili and Soy (麻婆豆腐) & Braised Whelk with Brown Sauce (红烧海螺) \\ \hline
    Tea-leaf and Camphor Smoked Duck (樟茶鸭) & Fried Pork Joint (锅烧肘子) \\ \hline
    Sauted Chicken Cubes with Chilli and Peanuts (宫保鸡丁) & Fried Tofu with Egg Wrapping (锅塌豆腐) \\ \hline
    Stewed Bean Curd with Minced Pork in Pepper Sauce (鱼香肉丝) & Quick-fried Mutton with Green Onion (葱爆羊肉) \\ \hline
    ~ & ~ \\ \hline
    \textbf{Huaiyang} & \textbf{Yuemin} \\ \hline
    Braised Shredded Chicken with Ham and Dried Tofu (大煮干丝) & Soup Simmered inside a Whole Winter Melon (冬瓜盅) \\ \hline
    Sweet and Sour Mandarin Fish (松鼠桂鱼) & Fried Pork with Pineapple (菠萝古老肉) \\ \hline
    Beggar's Chicken (叫花鸡) & Steamed Rice with Dried Duck (腊味煲仔饭) \\ \hline
    The Lion’s Head (the Meatball) in Yangzhou Style (淮扬狮子头) & Buddha Jumps over the Wall (佛跳墙) \\ \hline
    
    \multicolumn{2}{c}{\textbf{Relevant Terminology}} \\ \hline
    The Staple Food & 主食 \\ \hline
    The Flavors & 风味 \\ \hline
    The Nutrition Structure & 饮食结构 \\ \hline
    Wheaten food & 面食 \\ \hline
    vegetarian diet & 素食 \\ \hline
    household dishes & 家常菜 \\ \hline
    complete presence of color, aroma and taste & 色香味有严格的要求 \\ \hline
    street food & 街边小吃 \\ \hline
    
    \multicolumn{2}{c}{\textbf{Adjectives for Dish Names}} \\ \hline
    skillful and delightful & 出神入化 \\ \hline
    suits both refined and popular tastes & 雅俗共赏 \\ \hline
    
    \multicolumn{2}{c}{\textbf{The Staple Food}} \\ \hline
    Sichuan & ~ \\ \hline
    Rice & 米饭 \\ \hline
    Rice Porridge & 粥 \\ \hline
    Rice Cake & 米糕 \\ \hline
    Steamed Vermicelli Roll & 肠粉 \\ \hline
    Guilin Rice Noodles & 桂林米粉 \\ \hline
    Stir-fried Rice Noodles with Beef & 干炒牛河 \\ \hline
    GuiYang Spring Roll & 丝娃娃 \\ \hline
    red oil hot pot & 红油火锅 \\ \hline
    Fish Filets in Hot Chilli Oil & 水煮鱼 \\ \hline
    spicy hot pot/Malatang & 麻辣烫 \\ \hline
    hot pot & 火锅 \\ \hline
    chuanchuan & 串串 \\ \hline
    ~ & ~ \\ \hline
    Northeast of China & ~ \\ \hline
    pickled cabbage & 酸菜 \\ \hline
    Fat Pork with Pickled Cabbage & 酸菜白肉 \\ \hline
    Pork and Pickled Cabbage dumplings & 猪肉酸菜水饺 \\ \hline
    ~ & ~ \\ \hline
    south of China & ~ \\ \hline
    Sweet and Sour Mandarin Fish & 松鼠桂鱼 \\ \hline
    Square Pork with sauce & 酱方 \\ \hline
    Fresh Pork and Salted Pork with Bamboo Shoot Soup & 腌笃鲜 \\ \hline
    Osmanthus Jelly & 桂花糕 \\ \hline
    ~ & ~ \\ \hline
    meat & ~ \\ \hline
    braised pork & 红烧肉 \\ \hline
    ~ & ~ \\ \hline
    Candied Sweet Potato & 拔丝地瓜 \\ \hline
    Fish in Sour Soup & 酸汤鱼 \\ \hline
    Salt Baked Razor Clam & 盐焗蛏子 \\ \hline
    Chicken Cubes with Chili Peppers & 辣子鸡丁 \\ \hline
    Balsam Pear Scrambled eggs & 苦瓜炒蛋 \\ \hline
    ~ & ~ \\ \hline
    fried rice & 炒饭 \\ \hline
    fried noodles & 炒面 \\ \hline
    Beijing roast duck & 北京烤鸭 \\ \hline
    Xiao long bao & 小笼包 \\ \hline
    Kung Pao Chicken & 宫保鸡丁 \\ \hline
    Boluo gulao rou (Sweet and sour pork with pineapple) & 菠萝古老肉 \\ \hline
    Fried shrimps with cashew nuts & 腰果炒虾仁 \\ \hline
    Stir-fried tofu in hot sauce/Ma po doufu & 麻婆豆腐 \\ \hline
    ~ & ~ \\ \hline
    Dongpo soup & 东坡羹 \\ \hline
    Dongpo Pork & 东坡肉 \\ \hline
    ~ & ~ \\ \hline
    Lamb Spine Hot Pot & 羊蝎子 \\ \hline
    ~ & ~ \\ \hline
    cornmeal pancake & 玉米饼 \\ \hline
    
    \multicolumn{2}{c}{\textbf{The Flavors}} \\ \hline
    Sichaun & spicy \\ \hline
    Northeast of China & sour \\ \hline
    south of China & sweetness \\ \hline
    ~ & ~ \\ \hline
    bitter & 苦 \\ \hline
    salty & 咸 \\ \hline
    ~ & ~ \\ \hline
    turns out to have a sour & 酸味绵柔 \\ \hline
    tender and crispy taste & 松软清脆 \\ \hline
    crispy outside and tender inside & 外脆里嫩 \\ \hline
    provides a sour and sweet taste & 酸甜可口 \\ \hline
    pungent & 刺激性的 \\ \hline
    fragrant & 清香的 \\ \hline
    mellow & 醇美的 \\ \hline
    Juicy & 多汁的 \\ \hline
    Greasy & 油腻 \\ \hline
    
    \multicolumn{2}{c}{\textbf{Wheaten Food}} \\ \hline
    north & ~ \\ \hline
    Dumplings & 饺子 \\ \hline
    Fried Noodles & 炒面 \\ \hline
    steamed Bings & 包子 \\ \hline
    Shandong Pancake & 山东煎饼 \\ \hline
    Fried Dumplings & 锅贴 \\ \hline
    south & ~ \\ \hline
    shao-mai & 烧麦 \\ \hline
    spring rolls & 春卷 \\ \hline
    fried dough sticks & 油条 \\ \hline
    
    \multicolumn{2}{c}{\textbf{Ingredients}} \\ \hline
    pepper & 辣椒 \\ \hline
    cabbage & 大白菜 \\ \hline
    prickly ash & 花椒 \\ \hline
    mandarin fish & 桂鱼 \\ \hline
    Wheat & 小麦 \\ \hline
    ~ & ~ \\ \hline
    meat & ~ \\ \hline
    pork & 猪肉 \\ \hline
    fish & 鱼肉 \\ \hline
    chicken & 鸡肉 \\ \hline
    duck & 鸭肉 \\ \hline
    ~ & ~ \\ \hline
    vegetarian food & ~ \\ \hline
    water shield & 茆(莼菜) \\ \hline
    lentil & 荏菽(扁豆) \\ \hline
    day lily & 谖草 (黄花菜) \\ \hline
    mulberry & 桑葚 \\ \hline
    ~ & ~ \\ \hline
    the spine of the lamb & 羊脊骨 \\ \hline
    ~ & ~ \\ \hline
    Mild green chillies & 青椒 \\ \hline
    ginger & 生姜 \\ \hline
    garlic & 大蒜 \\ \hline
    aubergine & 茄子 \\ \hline
    cucumber & 黄瓜 \\ \hline
    peanut & 花生 \\ \hline
    ~ & ~ \\ \hline
    a piece of flour pancake & 一张面皮 \\ \hline
    
    \multicolumn{2}{c}{\textbf{Seasonings}} \\ \hline
    salt & 盐 \\ \hline
    red sauce & 酱油 \\ \hline
    dark soy sauce & 黑酱油/老抽 \\ \hline
    
    \multicolumn{2}{c}{\textbf{Food Preparation Methods}} \\ \hline
    natural fermentation & 自然发酵 \\ \hline
    pickled & 腌制 \\ \hline
    braise & 炖 \\ \hline
    toss & 拌 \\ \hline
    chop & 剁 \\ \hline
    shred & 切 \\ \hline
    slice & 割 \\ \hline
    dice & 削 \\ \hline
    ~ & ~ \\ \hline
    cooking utensils & ~ \\ \hline
    cleaver & 菜刀 \\ \hline
    rolling pin & 擀面杖 \\ \hline
    
    \multicolumn{2}{c}{\textbf{Drinks}} \\ \hline
    Tea & 茶 \\ \hline
    
    \multicolumn{2}{c}{\textbf{Elegant Names for Dishes}} \\ \hline
    Appetizers combination & 八方宾客(富贵八小碟) \\ \hline
    Beggars chicken & 名扬天下(新派叫花鸡) \\ \hline
    Double-boiled duck with lotus seed & 大展宏图(鲜莲子炖老鸭) \\ \hline
    Braised vegetable with mushroom & 包罗万象(鲜鲍菇扒时蔬) \\ \hline
    Deep-fried prawn with almond & 紧密合作(杏仁大明虾) \\ \hline
    Braised seasonal vegetable, Beijing style & 风景如画(京扒扇形蔬) \\ \hline
    Pan-fried Australian beef with black pepper & 共谋发展(黑椒澳洲牛柳) \\ \hline
    Fried rice with minced beef & 携手共赢(生炒牛松饭) \\ \hline
    Roasted lamb chop with cumin & 干秋盛世(孜然烤羊排) \\ \hline
    Chinese petit fours & 共建和平(美点映双辉) \\ \hline
    Seasonal fresh fruit platter & 承载梦想(环球鲜果盆) \\ \hline
    ~ & ~ \\ \hline
    Ant climbing trees & 蚂蚁上树 \\ \hline
    Lion's head & 狮子头 \\ \hline
    Treasures filling the home & 金玉满堂 \\ \hline
    A flight of white egrets in blue sky & 一行白鹭上青天 \\ \hline
    The grasses up north are as blue as jade & 燕草如碧丝 \\ \hline
    A timely snow promising a good harvest & 瑞雪兆丰年 \\ \hline
    
\end{longtable}

\begin{question}{}{}
    Please prepare one menu for a foreign student who comes to China for the first time.
\end{question}

\begin{longtable}{|p{2cm}|p{2cm}|p{5cm}|p{5cm}|}
    \caption{Menu} \\
    \hline
    \multicolumn{4}{|c|}{\textbf{Varieties of dishes}} \\
    \hline
    \textbf{Varieties of dishes} & \textbf{Dish name} & \textbf{Dish description} & \textbf{Tips} \\
    \hline
    \endfirsthead
    
    \caption[]{(续上) Menu} \\
    \hline
    \multicolumn{4}{|c|}{\textbf{Varieties of dishes}} \\
    \hline
    \textbf{Varieties of dishes} & \textbf{Dish name} & \textbf{Dish description} & \textbf{Tips} \\
    \hline
    \endhead
    
    \hline
    \endfoot
    
    \hline
    \endlastfoot
    
    Meat Dishes & ~ & ~ & These dishes are not halal. Muslim patrons may wish to choose other options. \\ \hline
    ~ & General Tso's Chicken(左宗棠鸡) & A popular chicken dish in American-Chinese cuisine known for its sweet and sour flavors, typically served with vegetables and sauce. & ~ \\ \hline
    ~ & Orange Chicken(橙子鸡) & A beloved Chinese dish featuring crispy chicken pieces coated in flour and fried until golden brown, then smothered in a tangy orange sauce. & ~ \\ \hline
    ~ & Black Pepper Chicken(黑胡椒鸡) & A stir-fried chicken dish seasoned with black pepper, usually paired with onions and other vegetables, characterized by its spicy and robust black pepper aroma. & Spiciness level can be adjusted. \\ \hline
    ~ & Szechuan Beef(四川牛肉) & A spicy Sichuan-style beef dish cooked with chili peppers and Sichuan peppercorns, giving it a numbing and fiery flavor profile. & ~ \\ \hline
    ~ & Beef \& Broccoli(西兰花炒牛肉) & A classic combination of tender beef strips and fresh broccoli florets stir-fried in a savory sauce, commonly found in Chinese-American cuisine. & ~ \\ \hline
    Staple Foods & ~ & ~ & ~ \\ \hline
    ~ & fried rice(炒饭) & A versatile dish made by frying cooked rice with various ingredients such as vegetables, meat, and sometimes eggs, resulting in a flavorful and filling meal. & ~ \\ \hline
    ~ & scallion pancakes(葱油饼) & Flaky flatbreads filled with scallions and brushed with sesame oil, often served as a side dish or appetizer in Chinese cuisine. & ~ \\ \hline
    ~ & Soup Dumplings (汤饺) & Steamed dumplings filled with juicy pork and broth, typically enjoyed by sipping the soup inside before eating the dough and filling. & ~ \\ \hline
    ~ & Shrimp Kung Pao Noodles(宫保虾粉) & A spicy noodle dish featuring shrimp cooked in a Kung Pao sauce, which includes peanuts, chili peppers, and Sichuan peppercorns. & Caution: Contains seafood. Not recommended for those with seafood allergies. \\ \hline
    ~ & Pan-Fried Noodles in Superior Soy Sauce(酱油炒面) & Wok-tossed noodles mixed with a rich soy-based sauce, creating a deliciously savory and slightly sweet flavor. & ~ \\ \hline
    Soups and Beverages & ~ & ~ & ~ \\ \hline
    ~ & wonton soup(馄饨汤) & A comforting soup made with small, stuffed wontons, typically filled with pork or shrimp, and served in a clear broth with vegetables. & ~ \\ \hline
    ~ & Tea(茶) & A wide range of tea varieties, from green to black, oolong to white, each offering unique flavors and health benefits. & ~ \\ \hline
    ~ & Rice Wine(米酒) & An alcoholic beverage made from fermented rice, often consumed as a digestif or used in cooking to add depth and complexity to dishes. & ~ \\ \hline
    Vegetarian Dishes & ~ & ~ & ~ \\ \hline
    ~ & Bang Bang Broccoli(西兰花) & A spicy and sweet dish featuring broccoli florets tossed in a bang bang sauce, a blend of sweet chili sauce, mayonnaise, and other seasonings. & ~ \\ \hline
    ~ & Candied Sweet Potato(拔丝地瓜) & A dessert dish where sweet potatoes are deep-fried and coated in a caramelized sugar glaze, providing a crunchy exterior and soft interior. & ~ \\ \hline
    ~ & cornmeal pancake(玉米饼) & A flatbread made from cornmeal, often served as a side dish or used as a wrap for various fillings. & ~ \\ \hline
    ~ & Soup Simmered inside a Whole Winter Melon(冬瓜盅) & A hearty soup prepared by simmering winter melon rind with various meats and vegetables, resulting in a fragrant and nourishing broth. & ~ \\ \hline
    ~ & The grasses up north are as blue as jade(燕草如碧丝) & A poetic reference to the lush green grasses in northern regions, often used metaphorically to describe beauty or abundance. & ~ \\ \hline
    Others & ~ & ~ & ~ \\ \hline
    ~ & egg rolls(蛋卷) & Crispy spring rolls filled with vegetables, meat, and sometimes noodles, typically served as an appetizer or snack. & Caution: Contains eggs. Not recommended for those with egg allergies. \\ \hline
    ~ & Osmanthus Jelly(桂花糕) & A delicate jelly-like dessert made with osmanthus flowers, often served chilled and garnished with syrup for a refreshing treat. & ~ \\ \hline
    ~ & Xiao long bao(小笼包) & Delicate steamed dumplings filled with pork and soup, best enjoyed by biting a small hole first to release the hot broth within. & Bite a small hole first to let the hot broth escape before consuming the entire dumpling. \\ \hline
    ~ & Square Pork with sauce(酱方) & A braised pork belly dish slow-cooked in a soy-based sauce, resulting in tender meat with a rich, savory flavor. & ~ \\ \hline
\end{longtable}


\section{Assignment-4}

\begin{question}{}{}
    Sum up the translation of Chinese idioms and culture-loaded expressions.
\end{question}

\begin{table}[!ht]
    \centering
    \caption{Idioms and Expressions}
    \begin{tabular}{|c|c|}
    \hline
    \textbf{English} & \textbf{Chinese} \\ \hline
        silence its voice & 以免它露出锋芒 \\ \hline
        have fallen in a coordinated attack & 遭到了里应外合的袭击 \\ \hline
        If I may, Your Imperial Majesty & 小人斗胆进言  皇帝陛下 \\ \hline
        be mine to kill & 被我手刃 \\ \hline
        Quiet Composed Graceful Elegant Poised Polite & 文静 沉着 优雅 贤淑 从容 有礼 \\ \hline
        Beautiful tool...for terrible work. & 此剑虽美,所行却恶 \\ \hline
        She is innocent of the world.Of men.And the evils of war. & 涉世未深,不知人心险恶,不知战争残酷 \\ \hline
        Her skin is white as milk. & 她肤如凝脂\\
Her fingers like the tender white roots of a green onion. & 指如葱白\\
Her eyes are like morning dewdrops… &目似朝露 \\ \hline
        cherry red lips & 樱桃小嘴 \\ \hline
        that donkey Yao & 姚那头蠢驴 \\ \hline
        You never know with women & 女人心  海底针 \\ \hline
        a small taste of what is to come & 小小甜头 \\ \hline
        riches will flow like a mighty river & 源源不断的金银财宝(财源滚滚) \\ \hline
        connect deeply to his chi & 人气合一 \\ \hline
        Tranquil as a forest but on fire within & 其徐如林 侵略如火 \\ \hline
        yields to force and redirects it & 避其锋芒  借力打力 \\ \hline
        Four ounces can move 1,000 pounds & 四两拨千斤 \\ \hline
        even though our training is not finished & 厉兵秣马犹未成 \\ \hline
        Your disgrace is worse than death & 你身负奇耻大辱  生不如死 \\ \hline
        I need your help & 请助我一臂之力 \\ \hline
        Rise up like a phoenix & 如凤凰般涅槃 \\ \hline
        One warrior knows another & 英雄识英雄 \\ \hline
        The green shoot has grown up to the sky & 稚嫩青葱已经长成参天大树 \\ \hline
        Over my shoulder & 蓦然一回首 \\ \hline
        All I know is that it's harder & 只知道阻且长 \\ \hline
    \end{tabular}
\end{table}

\begin{table}[!ht]
    \centering
    \caption{Clan-related Terms}
    \begin{tabular}{|c|c|}
    \hline
        \textbf{English} & \textbf{Chinese} \\ \hline
        brings honor & 给家里争光 \\ \hline
        Dishonor to the Hua family & 花家之耻 \\ \hline
        Ancestors & 先祖在上 \\ \hline
        ancestral guardian & 列祖列宗 \\ \hline
        family sword & 祖传的剑 \\ \hline
        Standing in my father's shoes & 继承父业 \\ \hline
    \end{tabular}
\end{table}

\begin{table}[!ht]
    \centering
    \caption{Marriage-related Expressions}
    \begin{tabular}{|c|c|}
    \hline
    \textbf{English} & \textbf{Chinese} \\ \hline
        matchmaker & 媒婆 \\ \hline
        matched & 定亲 \\ \hline
        married to one & 娶亲 \\ \hline
        an auspicious match & 如意郎君 \\ \hline
    \end{tabular}
\end{table}

\begin{table}[!ht]
    \centering
    \caption{Proper Noun}
    \begin{tabular}{|c|c|}
    \hline
    \textbf{English} & \textbf{Chinese} \\ \hline
        Mulan & 花木兰 \\ \hline
        chi & 气 \\ \hline
        witch & 巫婆 \\ \hline
        phoenix & 凤凰 \\ \hline
        is consumed by flame and emerges again & (凤凰)浴火重生 \\ \hline
        Silk Road & 丝绸之路 \\ \hline
        Rourans & 柔然人 \\ \hline
        Your Majesty & 陛下 \\ \hline
        Böri Khan & 鲍里可汗 \\ \hline
        Imperial Army & 中军 \\ \hline
        I think one was a male, one was a female. & 雄兔脚扑朔  雌兔眼迷离\\
        But you know, you can't really tell when they're running that fast. &双兔傍地走 安能辨我是雄雌 \\ \hline
        Imperial edict & 圣旨 \\ \hline
        conscription & 征兵令 \\ \hline
        three pillars of virtue & 三大品德 \\ \hline
        Loyal Brave True & 忠 勇 真 \\ \hline
        Sergeant & 都尉 \\ \hline
        New Palace & 新殿 \\ \hline
        Imperial guardsmen & 宿卫军 \\ \hline
        Chancellor & 宰相 \\ \hline
        Son of Heaven & 真龙天子 \\ \hline
        sons of the empire & 臣子 \\ \hline
        invitation & 册封 \\ \hline
    \end{tabular}
\end{table}

\begin{table}[!ht]
    \centering
    \caption{Four-character Idiom}
    \begin{tabular}{|c|c|}
    \hline
    \textbf{English} & \textbf{Chinese} \\ \hline
        A young shoot, all green… & 稚嫩青葱 \\ \hline
        blade & 刀光剑影 \\ \hline
        Take control of yourself! & 成何体统 \\ \hline
        allow this to continue & 放任不管 \\ \hline
        has been disrupted & 深受其扰 \\ \hline
        is on its knees & 跪地求饶 \\ \hline
        A scorned dog & 丧家之犬 \\ \hline
        bring honor to us all & 光宗耀祖 \\ \hline
        be silent & 不发一言 \\ \hline
        be invisible & 化若无形 \\ \hline
        Our land is at war & 战火四起 \\ \hline
        made a terrible mistake & 铸成大错 \\ \hline
        in great danger & 只身犯险 \\ \hline
        Disadvantage can be turned into an advantage & 扭转乾坤 \\ \hline
        moves first controls the enemy & 先发制人 \\ \hline
        changes nothing & 无济于事 \\ \hline
        Liar & 满口胡言 \\ \hline
        are an imposter & 冒名顶替 \\ \hline
        are just at the beginning of your power & 初露峥嵘 \\ \hline
        without question & 毋庸置疑 \\ \hline
        was impossible & 天方夜谭 \\ \hline
        take the noble path & 改邪归正 \\ \hline
        Every motion & 一举一动 \\ \hline
    \end{tabular}
\end{table}
\section{Assignment-5}

\begin{question}{}{}
    Based on knowledge of intercultural communication, write a report to analyze how Mulan successfully tells the story of Hua Mulan in China.
\end{question}

There are various reasons why \textit{Mulan} has succeeded in storytelling, and four main reasons stand out.

From the perspective of the \textbf{intercultural speaker}, \textit{Mulan} was directed by a foreign hand, interpreting Chinese culture through a Western lens, bringing a fresh approach imbued with Western values and culture, thus naturally resonating with Western audiences. For instance, in traditional Chinese perceptions, the filial piety embodied by Hua Mulan is a dominant theme. However, in the film \textit{Mulan}, the element of ``filial piety'' is somewhat diluted from Hua Mulan's character, instead focusing on her growth as a soldier and shaping her into a heroic figure and legend. The intercultural communicator has a profound understanding of Chinese culture and presents aspects that interest Western audiences, such as Chinese martial arts. Considering the Western audience's comprehension level, certain details have been ``translated'' to make them more relatable. For example, in the movie, Mulan uses an apple for sustenance during her journey, whereas historically, flatbreads would have been more common. However, the creators chose to use an apple because it is more familiar to Western viewers.

From the perspective of the \textbf{audience}, due to the curiosity Westerners have about Chinese martial arts, the film exaggerates the concept of \textbf{``Qi''} (vital energy) found in Chinese martial arts, making it surreal, and adds elements of \textbf{``magic''} popular in Western culture, making it more appealing. Additionally, because of the unfamiliarity Westerners have with China, the movie creates scenes with an exotic allure, such as \textbf{``earth buildings''}, \textbf{``Hanfu''} (traditional Han clothing), and \textbf{``yellow face makeup''}. Furthermore, the hero culture prevalent in the West is reflected in how Mulan is portrayed as a hero, aligning with their mainstream aesthetic preferences.

From the perspective of \textbf{strategies of communication}, \textit{Mulan} uses the medium of film to vividly convey its story, possessing market value and leveraging the fame of Hollywood to achieve extensive reach. The film focuses on Mulan as the protagonist, providing close-ups and smooth camera work, enhancing the viewing experience.

From the perspective of the \textbf{content of the communication}, the film not only spreads material culture like earth buildings and Hanfu but also spiritual culture, such as the traditional Chinese concept of \textbf{``filial piety.''} It emphasizes and adapts according to Western cultural preferences and aesthetics, for instance, downplaying the ``filial piety'' aspect while highlighting Mulan's heroic qualities. It also enriches the narrative with battle scenes to create a dramatic and engaging storyline, blending Western magical elements like \textbf{``magic''} with traditional Chinese elements like the phoenix and Chinese martial arts, thereby making it highly attractive.

Moreover, the story of \textit{Mulan} was already widely known in Western countries, bolstered by previous works, contributing to a broad potential market.



\section{Assignment-6}

\begin{question}{}{}
    Design a case to introduce Chinese festivals to your foreign friend who just arrives in Shanghai. (300-400words)
\end{question}

Welcome to \textbf{Shanghai}, \textbf{China}, and thank you for your interest in \textbf{Chinese festivals}. Now, let me introduce them to you. I hope this will be helpful.

Did you enjoy playing with water when you were a child? Are you familiar with a festival where water play is the main activity? As you may know, \textbf{China} is a country where many ethnic groups coexist harmoniously. Although I am \textbf{Han Chinese}, many ethnic groups also have varieties of festivals, among which are the \textbf{Tibetan New Year} of Tibetan people and the \textbf{Water Splashing Festival} of the Dai people. Taking the \textbf{Water Splashing Festival} of the Dai people as an example, this festival marks the Dai New Year and usually takes place in mid-April, coinciding with the hottest time of the year. During the festival, people splash water on each other as a form of cleansing and blessing. It is believed that the water washes away bad luck and purifies individuals for the coming year. Participants often use buckets, water guns, or simply their hands to splash water, and the streets become filled with joy and laughter as everyone gets involved.

Unlike the calendar system adopted in the United States, ancient China adopted a \textbf{lunisolar calendar}, and some festivals have the same numbers for the day and the month. For example, the \textbf{Spring Festival} on the first day of the first month, the \textbf{Dragon Boat Festival} on the fifth day of the fifth month, and the \textbf{Double Ninth Festival} on the ninth day of the ninth month. However, they are not the products of some number games. In the eyes of the ancient Chinese, these days were considered inauspicious, therefore, people would take a bath, drive off bad luck, offer sacrifices to gods, and pray for good luck. For instance, on the \textbf{Spring Festival}, people would clean the house, set off firecrackers and fireworks, decorate the house, stay up, and feast. On the \textbf{Dragon Boat Festival}, people would hang Chinese mugwort and calamus on doors, wear perfume pouches, and drink realgar wine.

In recent years, some new festivals have emerged in China, such as November 11, also known as the \textbf{Double 11 Festival}, which is similar to \textbf{Black Friday} in America. Additionally, the \textbf{China Shanghai International Arts Festival} is currently underway, and if you are interested, I welcome you to join me in exploring it.

\section{Assignment-7}

\begin{question}
    To review the video clips about Qingming and design a case to introduce Qingming to your foreign friend who just arrives in China.
\end{question}

\textbf{Welcome to Shanghai, China!} I heard you are very interested in China's \textbf{Qingming Festival}, so let me give you a brief introduction. I hope it will be helpful.

\textbf{Similar to your April Fool's Day}, Qingming Festival, also known as \textbf{Tomb Sweeping Day}, falls around April 4 or 5. It is a time for people to reflect and remember family members who have passed away by visiting their tombs to clean and pay respects. The festival has been observed by Chinese people for over 2,500 years, dating back to when emperors would hold ceremonies to honor their ancestors.

Do you enjoy outings or flying kites? These are also customs of the \textbf{Qingming Festival}. Today, people pay their respects by visiting and sweeping the graves of their relatives, leaving offerings, burning incense, and paper money. In recent years, people have been burning essentials of the afterlife, such as cars and iPhones. The day is also regarded as the start of spring, when people go outside to enjoy the blossoming spring and special Qingming foods like \textbf{green rice balls}. I personally love green rice balls and hope you will enjoy them too.

To truly understand this festival, we need to look at the \textbf{Cold Food Festival}. There is an interesting story about the Cold Food Festival. It memorializes \textbf{Jie Zitui}, who made a broth from his own flesh (\textit{incredible, isn't it?}) for Prince Chong'er during his exile. When the prince became king, he forgot about Jie and gave positions to other officials. Jie, not minding, moved to a remote mountain with his mother. When the king realized his mistake, he went to find Jie but couldn't persuade him to come down from the mountain. The king set fire to the mountain, and Jie was found dead under a tree with his mother on his back. The king ordered no fire to be set on this day to remember Jie's sacrifice, leading to the \textbf{Cold Food Festival}. Over time, Chinese ancestors extended these practices into \textbf{Qingming}, and people eventually combined them into a single day.

Spring is just a few months away. If you are interested, let's go kite flying at the \textbf{Shanghai International Kite Flying Field in Fengxian Bay}. We can then make green rice balls together.
\section{Assignment-8}

\begin{question}
    Learning Reflections on Class Presentations. (400 words)
\end{question}

I learned from this group's presentation how to analyze the reasons for the success of \textbf{"Black Myth: Wukong"} in intercultural communication. There are various reasons why the game \textbf{"Black Myth: Wukong"} has succeeded in intercultural communication, and four main reasons stand out.

From the perspective of the \textbf{intercultural speaker}, the developers of \textbf{"Black Myth: Wukong"}, Game Science, are deeply rooted in Chinese culture. They draw heavily from the classic Chinese novel \textbf{"Journey to the West"}, which is a well-known and beloved tale in China. By leveraging this rich cultural heritage, the developers act as effective intercultural speakers, bridging the gap between Chinese and global audiences. Their deep understanding of the source material allows them to create a game that resonates with local players while also appealing to international gamers through universal themes of \textbf{heroism, adventure, and mythology}.

From the perspective of the \textbf{audience}, the target audience for \textbf{"Black Myth: Wukong"} is diverse, encompassing both domestic and international players. Domestically, the game appeals to Chinese players who are familiar with the story of \textbf{"Journey to the West"} and can appreciate the intricate details and references. Internationally, the game attracts gamers who are interested in high-quality action-adventure games and are curious about Eastern mythology and storytelling. The developers have successfully catered to both groups by balancing cultural specificity with universal appeal, ensuring that the game is accessible and engaging for a wide range of players.

From the perspective of \textbf{strategies of communication}, Game Science employs several effective strategies to communicate the game's narrative and mechanics to a global audience. First, they use high-quality visuals and animations to bring the world of \textbf{"Journey to the West"} to life, making the game visually stunning and immersive. Second, they provide detailed lore and background information through in-game cutscenes and collectibles, allowing players to delve deeper into the story and characters. Additionally, the game features multiple language options, including \textbf{English}, which helps to break down language barriers and makes the game accessible to non-Chinese speaking players.

From the perspective of the \textbf{content of the communication}, the content of \textbf{"Black Myth: Wukong"} is rich and multifaceted. The game's narrative is deeply rooted in Chinese mythology, featuring iconic characters like \textbf{Sun Wukong (the Monkey King)} and \textbf{Xuanzang (the monk)}. The story is told through a combination of cinematic cutscenes, dialogue, and environmental storytelling, providing a comprehensive and engaging experience. The gameplay itself is designed to be challenging yet rewarding, with fluid combat mechanics and a variety of abilities that reflect the supernatural powers of the characters. This blend of cultural depth and high-quality gameplay ensures that the game is both culturally significant and entertaining.

In conclusion, \textbf{"Black Myth: Wukong"} has achieved success in intercultural communication by effectively positioning itself as a bridge between Chinese and global gaming communities. The developers' deep understanding of their cultural roots, combined with strategic communication and engaging content, has allowed the game to resonate with a diverse audience, making it a standout title in the international gaming market.

\chapter{Review Materials}
\section*{题型}
Part I Case Analysis(25).\\
Directions: Based on knowledge of intercultural communication, write a report to analyze how \textbf{Mulan} or \textbf{Li Ziqi} successfully tells the story.(300-400 words)↵\\
从四个角度,多用视频的字同句,字数可以超一点,到500字。\\
评分标准↵:\\
第四档 (18-25分)\\
1 覆盖了所有要点。\\
2 语言通顺较复杂, 有少许语法错误,有效使用了语句间的连接成分,全文结构紧凑。\\
第三档(10-17分)\\
1 覆盖了大部分要点。\\
2 语言基本通顺,有部分语法错误, 语句间的连接较为正确有效。\\
第二档(5-9分)\\
1 覆盖了部分要点。\\
2 语言不够通顺, 语法错误较多, 语句间连接不太有效。\\
第一档(5分以下).\\
1 覆盖要点较少。

Part II Case Practice.(25)↵\\
Directions: Design a case to introduce \textbf{Chinese food} or \textbf{Chinese festival} or \textbf{Qingming} to your foreign friend who just arrives in Shanghai.(200-300 words).\\
主要是Chinese food,Chinese festival,清明节,可达到400字。刚到上海的外国朋友,接收度,讲清楚是什么,怎么enjoy.节目介绍注意顺序。

Part III Translation.(25)↵\\
Directions: Translate the following passage into Chinese.

Part IV Translation,(25)↵\\
Directions: Translate the following passage into English.\\
翻译的标准:\\
1.正确,不错译,漏译,多译。\\
2.通顺,符合语法和表达习惯。

\section{Li Ziqi}

\begin{question}{}{}
    Based on knowledge of intercultural communication, write a report to analyze how \textbf{Li Ziqi} successfully tells the story.(300-400 words)
\end{question}

There are various reasons why \textbf{Li Ziqi} has succeeded in storytelling, and four main reasons stand out.

From the perspective of the \textbf{intercultural speaker}, as the protagonist of the videos, \textbf{Li Ziqi} is young and beautiful, exuding no sense of threat but rather an amiable charm that evokes affection. Moreover, \textbf{Li Ziqi} is well-versed in Chinese traditional culture and its dissemination, choosing themes with thoughtful deliberation. The \textbf{peach blossom}'s universal appeal and status as a distinct symbol of Chinese culture, free from political, religious, or racial biases, make it an ideal cultural medium. For instance, it would not be suitable to use beef or mutton as cultural carriers for audiences including vegetarians or Muslims. Additionally, \textbf{Li Ziqi} integrates aspects of daily life—clothing, food, shelter, and travel—into her narrative, silently showcasing the essence of Chinese traditional culture in the mundanity of daily life, such as the creation of \textbf{Hanfu}, \textbf{peach blossom rice balls}, \textbf{peach blossom rice wine}, and \textbf{peach blossom cakes}, presenting these elements in an accessible manner, as if saying, ``It is right there."

From the perspective of the \textbf{audience}, while Westerners often enjoy camping, Chinese people prefer dining out at restaurants. Thus, \textbf{Li Ziqi}'s decision to conclude her video with a picnic in the peach grove alongside her grandmother, transitioning the scene from a confined space to an expansive orchard, and ending with an aerial shot, achieves multiple purposes at once. Furthermore, given the high degree of modernization in Europe and America, the natural ambiance \textbf{Li Ziqi} presents serves as a refreshing oasis. This resonates with the audience’s psychology and expectations, endowing the content with the potential for widespread appeal.

From the perspective of \textbf{strategies of communication}, \textbf{Li Ziqi} employs short videos for intercultural communication, crafted with exquisite quality, appealing to a broad audience. These videos focus on situational storytelling, with \textbf{Li Ziqi} actively involved in demonstrating processes, largely without dialogue or third-party narration, thus avoiding linguistic barriers. Even when text descriptions are used, they are accompanied by vivid imagery, making them self-explanatory. Similar to silent films, this approach provides a shared language for viewers, awakening common cultural elements among them. Moreover, this ``immersive" experience leaves a lasting impression on viewers.

From the perspective of the \textbf{content of the communication}, \textbf{Li Ziqi} not only promotes the tangible elements of Chinese traditional culture, such as \textbf{Hanfu} clothing, traditional dwellings, and cuisine like \textbf{peach blossom rice balls}, \textbf{peach blossom rice wine}, and \textbf{peach blossom cakes}, but also conveys the intangible spirit of this culture. She silently communicates the beautiful familial bond between herself and her grandmother—a sentiment that transcends borders and is a universal human emotion, reflecting China's tradition of respecting the elderly.

In addition, factors such as \textbf{Li Ziqi}'s massive following and the current wave of Chinese traditional culture spreading to the world also contribute to the success.
\section{Mulan}

\begin{question}{}{}
    Based on knowledge of intercultural communication, write a report to analyze how \textbf{Mulan} successfully tells the story.(300-400 words)
\end{question}

There are various reasons why \textit{Mulan} has succeeded in storytelling, and four main reasons stand out.

From the perspective of the \textbf{intercultural speaker}, \textit{Mulan} was directed by a foreign hand, interpreting Chinese culture through a Western lens, bringing a fresh approach imbued with Western values and culture, thus naturally resonating with Western audiences. For instance, in traditional Chinese perceptions, the filial piety embodied by Hua Mulan is a dominant theme. However, in the film \textit{Mulan}, the element of ``filial piety'' is somewhat diluted from Hua Mulan's character, instead focusing on her growth as a soldier and shaping her into a heroic figure and legend. The intercultural communicator has a profound understanding of Chinese culture and presents aspects that interest Western audiences, such as Chinese martial arts. Considering the Western audience's comprehension level, certain details have been ``translated'' to make them more relatable. For example, in the movie, Mulan uses an apple for sustenance during her journey, whereas historically, flatbreads would have been more common. However, the creators chose to use an apple because it is more familiar to Western viewers.

From the perspective of the \textbf{audience}, due to the curiosity Westerners have about Chinese martial arts, the film exaggerates the concept of \textbf{``Qi''} (vital energy) found in Chinese martial arts, making it surreal, and adds elements of \textbf{``magic''} popular in Western culture, making it more appealing. Additionally, because of the unfamiliarity Westerners have with China, the movie creates scenes with an exotic allure, such as \textbf{``earth buildings''}, \textbf{``Hanfu''} (traditional Han clothing), and \textbf{``yellow face makeup''}. Furthermore, the hero culture prevalent in the West is reflected in how Mulan is portrayed as a hero, aligning with their mainstream aesthetic preferences.

From the perspective of \textbf{strategies of communication}, \textit{Mulan} uses the medium of film to vividly convey its story, possessing market value and leveraging the fame of Hollywood to achieve extensive reach. The film focuses on Mulan as the protagonist, providing close-ups and smooth camera work, enhancing the viewing experience.

From the perspective of the \textbf{content of the communication}, the film not only spreads material culture like earth buildings and Hanfu but also spiritual culture, such as the traditional Chinese concept of \textbf{``filial piety.''} It emphasizes and adapts according to Western cultural preferences and aesthetics, for instance, downplaying the ``filial piety'' aspect while highlighting Mulan's heroic qualities. It also enriches the narrative with battle scenes to create a dramatic and engaging storyline, blending Western magical elements like \textbf{``magic''} with traditional Chinese elements like the phoenix and Chinese martial arts, thereby making it highly attractive.

Moreover, the story of \textit{Mulan} was already widely known in Western countries, bolstered by previous works, contributing to a broad potential market.
\section{Chinese food}

\begin{question}{}{}
    Design a case to introduce \textbf{Chinese food} to your foreign friend who just arrives in Shanghai.(200-300 words).
\end{question}

Welcome to Shanghai, China! I'm glad to hear that you are interested in Chinese food. Overall, \textbf{Chinese cuisine is incredibly diverse}, offering a wide range of flavors and a balanced mix of meat and vegetables. I believe there will be something to suit your taste, but I would like to recommend a menu of Chinese dishes that I think you will enjoy.

The first dish I want to recommend is \textbf{fried rice}. It is made by frying rice and mixing it with various ingredients. You can choose what you want to add to your dish. If you prefer noodles instead of rice, we have \textbf{fried noodles}. If you enjoy a mix of different ingredients, you can try \textbf{Malatang}. It is a combination of your favorite vegetables and meat cooked in a delicious broth.

Since you are not Muslim, the next dish I recommend is \textbf{Beijing roast duck}. \textit{Beijing roast duck is tender and juicy with a rich flavor, but enjoying it properly requires a specific method.} The right way to eat Beijing roast duck is to take a piece of flour pancake, spread some sauce on it, and then add a small serving of the duck skin, meat, and vegetables. Roll everything up and enjoy the delicious taste. \textbf{Don't miss this dish!}

Next, I recommend a local Shanghai specialty, \textbf{shengjian}, also known as pan-fried bao. \textit{Pan-fried bao consists of small buns filled with pork and a burst of savory soup.} The buns are cooked in a flat pan with water and oil, resulting in a golden, crispy bottom and a soft, fluffy top. To enjoy pan-fried bao, use chopsticks or a fork to pick up the bun. Take a small bite at the top to make a hole, then sip the hot soup inside to prevent burning. After sipping the soup, eat the remaining bun, which has a crispy bottom and a soft, savory filling.

For drinks, you can try \textbf{tea}. You might have had tea in the United States, but it would be great to taste some authentic Chinese tea.

Additionally, if you are not a fan of spicy food, you can request a lower level of spiciness when ordering.

If you have the time, let's meet up and have Chinese food at \textbf{Hongji Plaza}, which is near Shanghai University.

\section{Chinese festivals}

\begin{question}{}{}
    Design a case to introduce \textbf{Chinese festival} to your foreign friend who just arrives in Shanghai.(200-300 words).
\end{question}

Welcome to \textbf{Shanghai}, \textbf{China}, and thank you for your interest in \textbf{Chinese festivals}. Now, let me introduce them to you. I hope this will be helpful.

Did you enjoy playing with water when you were a child? Are you familiar with a festival where water play is the main activity? As you may know, \textbf{China} is a country where many ethnic groups coexist harmoniously. Although I am \textbf{Han Chinese}, many ethnic groups also have varieties of festivals, among which are the \textbf{Tibetan New Year} of Tibetan people and the \textbf{Water Splashing Festival} of the Dai people. Taking the \textbf{Water Splashing Festival} of the Dai people as an example, this festival marks the Dai New Year and usually takes place in mid-April, coinciding with the hottest time of the year. During the festival, people splash water on each other as a form of cleansing and blessing. It is believed that the water washes away bad luck and purifies individuals for the coming year. Participants often use buckets, water guns, or simply their hands to splash water, and the streets become filled with joy and laughter as everyone gets involved.

Unlike the calendar system adopted in the United States, ancient China adopted a \textbf{lunisolar calendar}, and some festivals have the same numbers for the day and the month. For example, the \textbf{Spring Festival} on the first day of the first month, the \textbf{Dragon Boat Festival} on the fifth day of the fifth month, and the \textbf{Double Ninth Festival} on the ninth day of the ninth month. However, they are not the products of some number games. In the eyes of the ancient Chinese, these days were considered inauspicious, therefore, people would take a bath, drive off bad luck, offer sacrifices to gods, and pray for good luck. For instance, on the \textbf{Spring Festival}, people would clean the house, set off firecrackers and fireworks, decorate the house, stay up, and feast. On the \textbf{Dragon Boat Festival}, people would hang Chinese mugwort and calamus on doors, wear perfume pouches, and drink realgar wine.

In recent years, some new festivals have emerged in China, such as November 11, also known as the \textbf{Double 11 Festival}, which is similar to \textbf{Black Friday} in America. Additionally, the \textbf{China Shanghai International Arts Festival} is currently underway, and if you are interested, I welcome you to join me in exploring it.
\section{Qingming}

\begin{question}
    Design a case to introduce \textbf{Qingming} to your foreign friend who just arrives in Shanghai.(200-300 words).
\end{question}

\textbf{Welcome to Shanghai, China!} I heard you are very interested in China's \textbf{Qingming Festival}, so let me give you a brief introduction. I hope it will be helpful.

\textbf{Similar to your April Fool's Day}, Qingming Festival, also known as \textbf{Tomb Sweeping Day}, falls around April 4 or 5. It is a time for people to reflect and remember family members who have passed away by visiting their tombs to clean and pay respects. The festival has been observed by Chinese people for over 2,500 years, dating back to when emperors would hold ceremonies to honor their ancestors.

Do you enjoy outings or flying kites? These are also customs of the \textbf{Qingming Festival}. Today, people pay their respects by visiting and sweeping the graves of their relatives, leaving offerings, burning incense, and paper money. In recent years, people have been burning essentials of the afterlife, such as cars and iPhones. The day is also regarded as the start of spring, when people go outside to enjoy the blossoming spring and special Qingming foods like \textbf{green rice balls}. I personally love green rice balls and hope you will enjoy them too.

To truly understand this festival, we need to look at the \textbf{Cold Food Festival}. There is an interesting story about the Cold Food Festival. It memorializes \textbf{Jie Zitui}, who made a broth from his own flesh (\textit{incredible, isn't it?}) for Prince Chong'er during his exile. When the prince became king, he forgot about Jie and gave positions to other officials. Jie, not minding, moved to a remote mountain with his mother. When the king realized his mistake, he went to find Jie but couldn't persuade him to come down from the mountain. The king set fire to the mountain, and Jie was found dead under a tree with his mother on his back. The king ordered no fire to be set on this day to remember Jie's sacrifice, leading to the \textbf{Cold Food Festival}. Over time, Chinese ancestors extended these practices into \textbf{Qingming}, and people eventually combined them into a single day.

Spring is just a few months away. If you are interested, let's go kite flying at the \textbf{Shanghai International Kite Flying Field in Fengxian Bay}. We can then make green rice balls together.
\section{English to Chinese and Chinese to English}

\begin{part}{}{}
    Also, the Qingming Festival is another day of family reunion. The alive visit the tomb of the dead and pay their respect. The Qingming Festival is the only one of the 24 solar terms that remains a festival. What's more, the tomb sweeping on the day is of national wide, honoring both close relatives and the ancestors. Chinese people hold the value of remembering the past while hoping for a good future, accepting mortality with optimism towards life. In the Qingming Festival, we also go spring outings, enjoying the tender sunshine of spring.

    清明节也是家人团聚的日子,生者来到逝者的墓前,向逝者表达敬意。清明节是24节气中唯一一个在现代中国还保持着节日地位的。这一天的祭扫活动是全国性的,不但纪念近亲,也纪念先人。中国人非常注重追古怀远,同时也知死乐生。清明节也是踏青、享受明媚春光的日子。
\end{part}

\begin{part}{}{}
    China, home to one in five of the planet's population, is the superpower of the world fears, but few really know. Ken Hom is the Godfather of Chinese food. He introduced the wok to the West more than 30 years ago. We are going to take a once-in-a-life adventure across China through food, to delve into its heart and soul. Food is the best way to explore Chinese culture, because we really live to eat. It's an epic trip, 3000 miles from the megacities of the East to the forgotten villages of the Wild West. We'll uncover the familiar, the secret and the surprising, cook simple and delicious dishes, and reveal the secrets of China, old and new. It's like a journey that I've always dreamt about.

    中国,世界五分之一人口的家园,世界敬畏的超级大国,却鲜有人真正了解。谭荣辉是中国菜的教父,三十余年前,他把锅引入了西方。我们即将开始一段绝无仅有的旅行,借由饮食走访中国,探寻这个古老国家最深处的灵魂。饮食是了解中国文化最好的途径,因为我们为食而生。这是一场史诗般的旅程,横跨3000英里,从东部繁华的都市,到西部荒凉、被人遗忘的村落。我们探寻熟悉的味道、秘密以及惊喜,做一些简单却美味的菜肴,揭开中国过去以及现在的秘密。就像我梦境中一直出现的那场旅行。
\end{part}

\begin{part}{}{}
    In recent years, there are some new “festivals” in China. November 11th, also known as the Double Eleven Festival, is regarded as the Singles' Day because of the four characters of “1”. On the Double Elven Festival of 2009, Alibaba cooperated with multiple brands and held the first Double Eleven Shopping Festival on Taobao.com to give cheer to the singles, which swept among consumers, storming the online shopping spree for the very first time. Today, a lot of online shopping platforms and the brick-and-mortar shops will provide tempting discounts on their products when it comes to the Double Eleven. Single and married, old and young, men and women, people are fully engaged in the shopping spree on the Double Eleven Festival, pretty much similar to American “Black Friday”.

    近年来,中国出现了一些新的节日。11月11日,也被称为“双十一”,因为含有四个“1”,故也称“光棍节”。2009 年的光棍节,阿里巴巴公司为了让单身的人快乐起来,联合很多个品牌,在网上商城举办了大型的打折活动,受到了消费者的热烈欢迎。首次举办就引发了网络购物狂欢。现在,各网购平台和实体店也都做较大力度的促销活动。在双十一期间,无论是单身,还是有家有室,男女老少都在双十一这天疯狂购物,类似于美国的“黑色星期五”。
\end{part}

\begin{part}{}{}
    中国的众多少数民族也有丰富多彩的节日, 著名的有:藏族的藏历新年、 蒙古族的那达慕、傣族的泼水节、傈僳族的刀杆节、彝族的火把节、白族的三月节、信仰伊斯兰教的民族的古尔邦节等。

    Many ethnic groups also have varieties of festivals, among which are the Tibetan New Year of Tibetan people, the Naadam of the Mongolians, the Water Splashing Festival of the Dai people, the Knife-Pole Festival of Lisu people, the Firebrand Festival of the Yi people, March Fair of Bai people, the Corban Festival of the Islamic group.
\end{part}

\begin{part}{}{}
    春节要大扫除、打爆竹、用红色装饰房子、守岁。中国人对自然特别亲近, 尤其是对月亮。中秋节的月亮是一年中最大最圆的,“圆”象征团圆, 所以中秋节是家人团圆的日子。

    On the Spring Festival, people would clean the house, set off firecrackers and fireworks, decorate the house with reds and stay up late. Chinese people are very close to nature, especially with the moon. The moon on the Mid-Autumn Festival turns out to be the fullest of the year. And given that fullness in Chinese culture is a symbol of union, the Mid-Autumn Festival thus became the festival for family reunion.
\end{part}

\begin{part}{}{}
    在你身上,我能看到他那把剑的影子,但或许那些影响太过沉重了,你不能让你父亲的身影限制你,你得好好培养你的天赋。宇宙万物皆统一于气,人亦由气化生,但唯有赤子之心方能使人气合一,从而铸就伟大的战士,其徐如林,侵掠如火。

    In you, I see the shadow of his sword. Perhaps this shadow falls heavy on your shoulders. You can't allow your father's legacy to hold you back. You need to cultivate your gift. The chi pervades the universe and all living things. We are all born with it. But only the most true will connect deeply to his chi and become a great warrior. Tranquil as a forest but on fire within.
\end{part}
\section{Final Examination Paper}

\begin{problem}
    Based on knowledge of intercultural communication, write a report to analyze how \textbf{Mulan} or \textbf{Li Ziqi}  successfully tells the story.(300-400 words)
\end{problem}

\begin{problem}
    Design a case to introduce \textbf{Chinese food} or \textbf{Chinese festival} or \textbf{Qingming} to your foreign friend who just arrives in Shanghai.(200-300 words).
\end{problem}

\begin{problem}
    Translate the following passage into Chinese.
    \begin{enumerate}
        \item Also, the Qingming Festival is another day of family reunion. The alive visit the tomb of the dead and pay their respect. The Qingming Festival is the only one of the 24 solar terms that remains a festival. What's more, the tomb sweeping on the day is of national wide, honoring both close relatives and the ancestors. Chinese people hold the value of remembering the past while hoping for a good future, accepting mortality with optimism towards life. In the Qingming Festival, we also go spring outings, enjoying the tender sunshine of spring.
        \item China, home to one in five of the planet's population, is the superpower of the world fears, but few really know. Ken Hom is the Godfather of Chinese food. He introduced the wok to the West more than 30 years ago. We are going to take a once-in-a-life adventure across China through food, to delve into its heart and soul. Food is the best way to explore Chinese culture, because we really live to eat. It's an epic trip, 3000 miles from the megacities of the East to the forgotten villages of the Wild West. We'll uncover the familiar, the secret and the surprising, cook simple and delicious dishes, and reveal the secrets of China, old and new. It's like a journey that I've always dreamt about.
        \item In recent years, there are some new “festivals” in China. November 11th, also known as the Double Eleven Festival, is regarded as the Singles' Day because of the four characters of “1”. On the Double Elven Festival of 2009, Alibaba cooperated with multiple brands and held the first Double Eleven Shopping Festival on Taobao.com to give cheer to the singles, which swept among consumers, storming the online shopping spree for the very first time. Today, a lot of online shopping platforms and the brick-and-mortar shops will provide tempting discounts on their products when it comes to the Double Eleven. Single and married, old and young, men and women, people are fully engaged in the shopping spree on the Double Eleven Festival, pretty much similar to American “Black Friday”.
        \item Many ethnic groups also have varieties of festivals, among which are the Tibetan New Year of Tibetan people, the Naadam of the Mongolians, the Water Splashing Festival of the Dai people, the Knife-Pole Festival of Lisu people, the Firebrand Festival of the Yi people, March Fair of Bai people, the Corban Festival of the Islamic group.
        \item On the Spring Festival, people would clean the house, set off firecrackers and fireworks, decorate the house with reds and stay up late. Chinese people are very close to nature, especially with the moon. The moon on the Mid-Autumn Festival turns out to be the fullest of the year. And given that fullness in Chinese culture is a symbol of union, the Mid-Autumn Festival thus became the festival for family reunion.
        \item In you, I see the shadow of his sword. Perhaps this shadow falls heavy on your shoulders. You can't allow your father's legacy to hold you back. You need to cultivate your gift. The chi pervades the universe and all living things. We are all born with it. But only the most true will connect deeply to his chi and become a great warrior. Tranquil as a forest but on fire within.
    \end{enumerate}
\end{problem}

\begin{problem}
    Translate the following passage into English.
    \begin{enumerate}
        \item 清明节也是家人团聚的日子,生者来到逝者的墓前,向逝者表达敬意。清明节是24节气中唯一一个在现代中国还保持着节日地位的。这一天的祭扫活动是全国性的,不但纪念近亲,也纪念先人。中国人非常注重追古怀远,同时也知死乐生。清明节也是踏青、享受明媚春光的日子。
        \item 中国,世界五分之一人口的家园,世界敬畏的超级大国,却鲜有人真正了解。谭荣辉是中国菜的教父,三十余年前,他把锅引入了西方。我们即将开始一段绝无仅有的旅行,借由饮食走访中国,探寻这个古老国家最深处的灵魂。饮食是了解中国文化最好的途径,因为我们为食而生。这是一场史诗般的旅程,横跨3000英里,从东部繁华的都市,到西部荒凉、被人遗忘的村落。我们探寻熟悉的味道、秘密以及惊喜,做一些简单却美味的菜肴,揭开中国过去以及现在的秘密。就像我梦境中一直出现的那场旅行。
        \item 近年来,中国出现了一些新的节日。11月11日,也被称为“双十一”,因为含有四个“1”,故也称“光棍节”。2009 年的光棍节,阿里巴巴公司为了让单身的人快乐起来,联合很多个品牌,在网上商城举办了大型的打折活动,受到了消费者的热烈欢迎。首次举办就引发了网络购物狂欢。现在,各网购平台和实体店也都做较大力度的促销活动。在双十一期间,无论是单身,还是有家有室,男女老少都在双十一这天疯狂购物,类似于美国的“黑色星期五”。
        \item 中国的众多少数民族也有丰富多彩的节日, 著名的有:藏族的藏历新年、 蒙古族的那达慕、傣族的泼水节、傈僳族的刀杆节、彝族的火把节、白族的三月节、信仰伊斯兰教的民族的古尔邦节等。
        \item 春节要大扫除、打爆竹、用红色装饰房子、守岁。中国人对自然特别亲近, 尤其是对月亮。中秋节的月亮是一年中最大最圆的,“圆”象征团圆, 所以中秋节是家人团圆的日子。
        \item 在你身上,我能看到他那把剑的影子,但或许那些影响太过沉重了,你不能让你父亲的身影限制你,你得好好培养你的天赋。宇宙万物皆统一于气,人亦由气化生,但唯有赤子之心方能使人气合一,从而铸就伟大的战士,其徐如林,侵掠如火。
    \end{enumerate}
\end{problem}

\end{document}
