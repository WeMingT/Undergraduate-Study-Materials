\section{Li Ziqi}

\begin{question}{}{}
    Based on knowledge of intercultural communication, write a report to analyze how \textbf{Li Ziqi} successfully tells the story.(300-400 words)
\end{question}

There are various reasons why \textbf{Li Ziqi} has succeeded in storytelling, and four main reasons stand out.

From the perspective of the \textbf{intercultural speaker}, as the protagonist of the videos, \textbf{Li Ziqi} is young and beautiful, exuding no sense of threat but rather an amiable charm that evokes affection. Moreover, \textbf{Li Ziqi} is well-versed in Chinese traditional culture and its dissemination, choosing themes with thoughtful deliberation. The \textbf{peach blossom}'s universal appeal and status as a distinct symbol of Chinese culture, free from political, religious, or racial biases, make it an ideal cultural medium. For instance, it would not be suitable to use beef or mutton as cultural carriers for audiences including vegetarians or Muslims. Additionally, \textbf{Li Ziqi} integrates aspects of daily life—clothing, food, shelter, and travel—into her narrative, silently showcasing the essence of Chinese traditional culture in the mundanity of daily life, such as the creation of \textbf{Hanfu}, \textbf{peach blossom rice balls}, \textbf{peach blossom rice wine}, and \textbf{peach blossom cakes}, presenting these elements in an accessible manner, as if saying, ``It is right there."

From the perspective of the \textbf{audience}, while Westerners often enjoy camping, Chinese people prefer dining out at restaurants. Thus, \textbf{Li Ziqi}'s decision to conclude her video with a picnic in the peach grove alongside her grandmother, transitioning the scene from a confined space to an expansive orchard, and ending with an aerial shot, achieves multiple purposes at once. Furthermore, given the high degree of modernization in Europe and America, the natural ambiance \textbf{Li Ziqi} presents serves as a refreshing oasis. This resonates with the audience’s psychology and expectations, endowing the content with the potential for widespread appeal.

From the perspective of \textbf{strategies of communication}, \textbf{Li Ziqi} employs short videos for intercultural communication, crafted with exquisite quality, appealing to a broad audience. These videos focus on situational storytelling, with \textbf{Li Ziqi} actively involved in demonstrating processes, largely without dialogue or third-party narration, thus avoiding linguistic barriers. Even when text descriptions are used, they are accompanied by vivid imagery, making them self-explanatory. Similar to silent films, this approach provides a shared language for viewers, awakening common cultural elements among them. Moreover, this ``immersive" experience leaves a lasting impression on viewers.

From the perspective of the \textbf{content of the communication}, \textbf{Li Ziqi} not only promotes the tangible elements of Chinese traditional culture, such as \textbf{Hanfu} clothing, traditional dwellings, and cuisine like \textbf{peach blossom rice balls}, \textbf{peach blossom rice wine}, and \textbf{peach blossom cakes}, but also conveys the intangible spirit of this culture. She silently communicates the beautiful familial bond between herself and her grandmother—a sentiment that transcends borders and is a universal human emotion, reflecting China's tradition of respecting the elderly.

In addition, factors such as \textbf{Li Ziqi}'s massive following and the current wave of Chinese traditional culture spreading to the world also contribute to the success.