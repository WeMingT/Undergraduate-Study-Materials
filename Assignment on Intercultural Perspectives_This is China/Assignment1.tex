\section{Assignment-1}

\begin{question}{}{}
    To make some comments on the challenges for China's intercultural communication in English(200-300 words).
\end{question}

In China's intercultural communication, there are primarily three challenges:

\textbf{Challenge 1}: \textbf{Language Barrier}

Aside from a few countries like Japan, which share a similar writing system with China, the languages of most countries greatly differ from Chinese. The high cost of learning and the difficulty of disseminating the language necessitate ``translation" as a means of intercultural communication, requiring additional human and financial resources. However, in countries where English is the official language, such a ``language barrier" does not exist.

Taking the classic work of Chinese traditional culture, the ``Tao Te Ching," as an example, its original Chinese text is written in classical Chinese, which is already challenging for modern Chinese to understand, let alone for foreign friends. The English translations of the ``Tao Te Ching" also face inevitable translation issues. For instance, the core term ``Tao" has been translated as ``way," ``dao," ``tao," among others, but it is undeniable that it is difficult to capture the profound essence of the Chinese character \textbf{道} through these translations.

\textbf{Challenge 2}: \textbf{Cultural Conflict}

When Chinese culture conflicts significantly with another culture in certain aspects, intercultural communication faces tremendous challenges. These challenges mainly stem from the resistance and skepticism of the inherent beliefs and aesthetic paradigms of that culture. For example, in some Middle Eastern countries where women wear veils and dress in black, colorful Chinese women's attire might encounter resistance and skepticism from religious customs. Similarly, while the mainstream aesthetic for men in the West is ``masculine," characterized by height and strength, in China, it is ``handsome," favoring slimness. Another example is the cultural differences between the East and the West in terms of family and filial piety depicted in the movie ``Hi, Mom.''

\textbf{Challenge 3}: \textbf{Political Status, Standpoint, and Policies}

Undeniably, in terms of political status, if China's political status is low, even with its splendid culture, it would still struggle with intercultural communication. After all, the burning of the Yuanming Yuan by the Eight-Nation Alliance was not intercultural communication but cultural destruction. Moreover, taking the Qing Dynasty as an example, its policy of resisting foreign cultures and implementing isolationism towards the end led to reciprocal resistance towards Chinese culture in other countries. This poses an insurmountable barrier for intercultural communication.
