\section{Assignment-8}

\begin{question}
    Learning Reflections on Class Presentations. (400 words)
\end{question}

I learned from this group's presentation how to analyze the reasons for the success of \textbf{"Black Myth: Wukong"} in intercultural communication. There are various reasons why the game \textbf{"Black Myth: Wukong"} has succeeded in intercultural communication, and four main reasons stand out.

From the perspective of the \textbf{intercultural speaker}, the developers of \textbf{"Black Myth: Wukong"}, Game Science, are deeply rooted in Chinese culture. They draw heavily from the classic Chinese novel \textbf{"Journey to the West"}, which is a well-known and beloved tale in China. By leveraging this rich cultural heritage, the developers act as effective intercultural speakers, bridging the gap between Chinese and global audiences. Their deep understanding of the source material allows them to create a game that resonates with local players while also appealing to international gamers through universal themes of \textbf{heroism, adventure, and mythology}.

From the perspective of the \textbf{audience}, the target audience for \textbf{"Black Myth: Wukong"} is diverse, encompassing both domestic and international players. Domestically, the game appeals to Chinese players who are familiar with the story of \textbf{"Journey to the West"} and can appreciate the intricate details and references. Internationally, the game attracts gamers who are interested in high-quality action-adventure games and are curious about Eastern mythology and storytelling. The developers have successfully catered to both groups by balancing cultural specificity with universal appeal, ensuring that the game is accessible and engaging for a wide range of players.

From the perspective of \textbf{strategies of communication}, Game Science employs several effective strategies to communicate the game's narrative and mechanics to a global audience. First, they use high-quality visuals and animations to bring the world of \textbf{"Journey to the West"} to life, making the game visually stunning and immersive. Second, they provide detailed lore and background information through in-game cutscenes and collectibles, allowing players to delve deeper into the story and characters. Additionally, the game features multiple language options, including \textbf{English}, which helps to break down language barriers and makes the game accessible to non-Chinese speaking players.

From the perspective of the \textbf{content of the communication}, the content of \textbf{"Black Myth: Wukong"} is rich and multifaceted. The game's narrative is deeply rooted in Chinese mythology, featuring iconic characters like \textbf{Sun Wukong (the Monkey King)} and \textbf{Xuanzang (the monk)}. The story is told through a combination of cinematic cutscenes, dialogue, and environmental storytelling, providing a comprehensive and engaging experience. The gameplay itself is designed to be challenging yet rewarding, with fluid combat mechanics and a variety of abilities that reflect the supernatural powers of the characters. This blend of cultural depth and high-quality gameplay ensures that the game is both culturally significant and entertaining.

In conclusion, \textbf{"Black Myth: Wukong"} has achieved success in intercultural communication by effectively positioning itself as a bridge between Chinese and global gaming communities. The developers' deep understanding of their cultural roots, combined with strategic communication and engaging content, has allowed the game to resonate with a diverse audience, making it a standout title in the international gaming market.