\section{Assignment-5}

\begin{question}{}{}
    Based on knowledge of intercultural communication, write a report to analyze how Mulan successfully tells the story of Hua Mulan in China.
\end{question}

There are various reasons why \textit{Mulan} has succeeded in storytelling, and four main reasons stand out.

From the perspective of the \textbf{intercultural speaker}, \textit{Mulan} was directed by a foreign hand, interpreting Chinese culture through a Western lens, bringing a fresh approach imbued with Western values and culture, thus naturally resonating with Western audiences. For instance, in traditional Chinese perceptions, the filial piety embodied by Hua Mulan is a dominant theme. However, in the film \textit{Mulan}, the element of ``filial piety'' is somewhat diluted from Hua Mulan's character, instead focusing on her growth as a soldier and shaping her into a heroic figure and legend. The intercultural communicator has a profound understanding of Chinese culture and presents aspects that interest Western audiences, such as Chinese martial arts. Considering the Western audience's comprehension level, certain details have been ``translated'' to make them more relatable. For example, in the movie, Mulan uses an apple for sustenance during her journey, whereas historically, flatbreads would have been more common. However, the creators chose to use an apple because it is more familiar to Western viewers.

From the perspective of the \textbf{audience}, due to the curiosity Westerners have about Chinese martial arts, the film exaggerates the concept of \textbf{``Qi''} (vital energy) found in Chinese martial arts, making it surreal, and adds elements of \textbf{``magic''} popular in Western culture, making it more appealing. Additionally, because of the unfamiliarity Westerners have with China, the movie creates scenes with an exotic allure, such as \textbf{``earth buildings''}, \textbf{``Hanfu''} (traditional Han clothing), and \textbf{``yellow face makeup''}. Furthermore, the hero culture prevalent in the West is reflected in how Mulan is portrayed as a hero, aligning with their mainstream aesthetic preferences.

From the perspective of \textbf{strategies of communication}, \textit{Mulan} uses the medium of film to vividly convey its story, possessing market value and leveraging the fame of Hollywood to achieve extensive reach. The film focuses on Mulan as the protagonist, providing close-ups and smooth camera work, enhancing the viewing experience.

From the perspective of the \textbf{content of the communication}, the film not only spreads material culture like earth buildings and Hanfu but also spiritual culture, such as the traditional Chinese concept of \textbf{``filial piety.''} It emphasizes and adapts according to Western cultural preferences and aesthetics, for instance, downplaying the ``filial piety'' aspect while highlighting Mulan's heroic qualities. It also enriches the narrative with battle scenes to create a dramatic and engaging storyline, blending Western magical elements like \textbf{``magic''} with traditional Chinese elements like the phoenix and Chinese martial arts, thereby making it highly attractive.

Moreover, the story of \textit{Mulan} was already widely known in Western countries, bolstered by previous works, contributing to a broad potential market.


