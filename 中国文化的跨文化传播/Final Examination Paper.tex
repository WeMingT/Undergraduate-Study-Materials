\section{Final Examination Paper}

\begin{problem}
    Based on knowledge of intercultural communication, write a report to analyze how \textbf{Mulan} or \textbf{Li Ziqi}  successfully tells the story.(300-400 words)
\end{problem}

\begin{problem}
    Design a case to introduce \textbf{Chinese food} or \textbf{Chinese festival} or \textbf{Qingming} to your foreign friend who just arrives in Shanghai.(200-300 words).
\end{problem}

\begin{problem}
    Translate the following passage into Chinese.
    \begin{enumerate}
        \item Also, the Qingming Festival is another day of family reunion. The alive visit the tomb of the dead and pay their respect. The Qingming Festival is the only one of the 24 solar terms that remains a festival. What's more, the tomb sweeping on the day is of national wide, honoring both close relatives and the ancestors. Chinese people hold the value of remembering the past while hoping for a good future, accepting mortality with optimism towards life. In the Qingming Festival, we also go spring outings, enjoying the tender sunshine of spring.
        \item China, home to one in five of the planet's population, is the superpower of the world fears, but few really know. Ken Hom is the Godfather of Chinese food. He introduced the wok to the West more than 30 years ago. We are going to take a once-in-a-life adventure across China through food, to delve into its heart and soul. Food is the best way to explore Chinese culture, because we really live to eat. It's an epic trip, 3000 miles from the megacities of the East to the forgotten villages of the Wild West. We'll uncover the familiar, the secret and the surprising, cook simple and delicious dishes, and reveal the secrets of China, old and new. It's like a journey that I've always dreamt about.
        \item In recent years, there are some new “festivals” in China. November 11th, also known as the Double Eleven Festival, is regarded as the Singles' Day because of the four characters of “1”. On the Double Elven Festival of 2009, Alibaba cooperated with multiple brands and held the first Double Eleven Shopping Festival on Taobao.com to give cheer to the singles, which swept among consumers, storming the online shopping spree for the very first time. Today, a lot of online shopping platforms and the brick-and-mortar shops will provide tempting discounts on their products when it comes to the Double Eleven. Single and married, old and young, men and women, people are fully engaged in the shopping spree on the Double Eleven Festival, pretty much similar to American “Black Friday”.
        \item Many ethnic groups also have varieties of festivals, among which are the Tibetan New Year of Tibetan people, the Naadam of the Mongolians, the Water Splashing Festival of the Dai people, the Knife-Pole Festival of Lisu people, the Firebrand Festival of the Yi people, March Fair of Bai people, the Corban Festival of the Islamic group.
        \item On the Spring Festival, people would clean the house, set off firecrackers and fireworks, decorate the house with reds and stay up late. Chinese people are very close to nature, especially with the moon. The moon on the Mid-Autumn Festival turns out to be the fullest of the year. And given that fullness in Chinese culture is a symbol of union, the Mid-Autumn Festival thus became the festival for family reunion.
        \item In you, I see the shadow of his sword. Perhaps this shadow falls heavy on your shoulders. You can't allow your father's legacy to hold you back. You need to cultivate your gift. The chi pervades the universe and all living things. We are all born with it. But only the most true will connect deeply to his chi and become a great warrior. Tranquil as a forest but on fire within.
    \end{enumerate}
\end{problem}

\begin{problem}
    Translate the following passage into English.
    \begin{enumerate}
        \item 清明节也是家人团聚的日子,生者来到逝者的墓前,向逝者表达敬意。清明节是24节气中唯一一个在现代中国还保持着节日地位的。这一天的祭扫活动是全国性的,不但纪念近亲,也纪念先人。中国人非常注重追古怀远,同时也知死乐生。清明节也是踏青、享受明媚春光的日子。
        \item 中国,世界五分之一人口的家园,世界敬畏的超级大国,却鲜有人真正了解。谭荣辉是中国菜的教父,三十余年前,他把锅引入了西方。我们即将开始一段绝无仅有的旅行,借由饮食走访中国,探寻这个古老国家最深处的灵魂。饮食是了解中国文化最好的途径,因为我们为食而生。这是一场史诗般的旅程,横跨3000英里,从东部繁华的都市,到西部荒凉、被人遗忘的村落。我们探寻熟悉的味道、秘密以及惊喜,做一些简单却美味的菜肴,揭开中国过去以及现在的秘密。就像我梦境中一直出现的那场旅行。
        \item 近年来,中国出现了一些新的节日。11月11日,也被称为“双十一”,因为含有四个“1”,故也称“光棍节”。2009 年的光棍节,阿里巴巴公司为了让单身的人快乐起来,联合很多个品牌,在网上商城举办了大型的打折活动,受到了消费者的热烈欢迎。首次举办就引发了网络购物狂欢。现在,各网购平台和实体店也都做较大力度的促销活动。在双十一期间,无论是单身,还是有家有室,男女老少都在双十一这天疯狂购物,类似于美国的“黑色星期五”。
        \item 中国的众多少数民族也有丰富多彩的节日, 著名的有:藏族的藏历新年、 蒙古族的那达慕、傣族的泼水节、傈僳族的刀杆节、彝族的火把节、白族的三月节、信仰伊斯兰教的民族的古尔邦节等。
        \item 春节要大扫除、打爆竹、用红色装饰房子、守岁。中国人对自然特别亲近, 尤其是对月亮。中秋节的月亮是一年中最大最圆的,“圆”象征团圆, 所以中秋节是家人团圆的日子。
        \item 在你身上,我能看到他那把剑的影子,但或许那些影响太过沉重了,你不能让你父亲的身影限制你,你得好好培养你的天赋。宇宙万物皆统一于气,人亦由气化生,但唯有赤子之心方能使人气合一,从而铸就伟大的战士,其徐如林,侵掠如火。
    \end{enumerate}
\end{problem}