\documentclass[a4paper,AutoFakeBold]{article}
\usepackage[UTF8]{ctex}

% 设置中文字体
\setCJKmainfont[BoldFont=SimHei, ItalicFont=KaiTi]{SimSun} % 主字体为宋体,粗体为黑体,斜体为楷体
\setCJKsansfont{SimHei} % 无衬线字体为黑体
\setCJKmonofont{FangSong} % 等宽字体为仿宋

% 设置西文字体
\setmainfont{Times New Roman}
\setsansfont{Arial}
\setmonofont{Courier New}

% 基础包
\usepackage{geometry}
\usepackage{setspace}
\usepackage{titlesec}
\usepackage{indentfirst}
\usepackage[breaklinks=true]{hyperref}
\usepackage{fancyhdr}
\usepackage[nottoc]{tocbibind}
\usepackage[perpage]{footmisc}
\usepackage{amsmath}
\usepackage{float}
\usepackage{graphicx}
\usepackage{caption}
\usepackage{booktabs}
\usepackage{array}
\usepackage{makecell}
\usepackage{pgfplots}
\pgfplotsset{compat=1.18}

% 设置图表标题格式
\captionsetup[table]{position=top, labelsep=quad, font={small}}    % 表格标题在上
\captionsetup[figure]{position=bottom, labelsep=quad, font={small}}    % 图片标题在下

% 定义资料来源命令
\newcommand{\source}[1]{%
    \vspace{2pt}%
    \par\noindent\small\kaishu 资料来源:#1%
    \vspace{6pt}%
}

% 页面设置
\geometry{left=3cm,right=3cm,top=2.5cm,bottom=2.5cm,headheight=15pt}

% 页眉页脚设置
\pagestyle{fancy}
\fancyhf{}
\renewcommand{\headrulewidth}{0.4pt}
\renewcommand{\footrulewidth}{0pt}

% 设置页眉样式
\fancyhead[C]{\songti\zihao{5} \textbf{论文标题}}
\fancyfoot[C]{\songti\zihao{5} 第 \thepage 页}

% 为目录页设置特殊页眉
\let\oldtableofcontents\tableofcontents
\renewcommand{\tableofcontents}{%
    \fancyhead[C]{\songti\zihao{5} \textbf{论文标题}}
    \oldtableofcontents
}

% 行距和段落间距设置
\setlength{\baselineskip}{20pt}
\setlength{\parskip}{6pt}

% 标题格式设置
\titleformat{\section}{\zihao{4}\bfseries\heiti}{\thesection}{1em}{}
\titleformat{\subsection}{\zihao{-4}\bfseries\heiti}{\thesubsection}{1em}{}

\begin{document}

% 标题页不显示页眉
\thispagestyle{empty}

% 标题
\begin{center}
    \zihao{2}\heiti\bfseries{论文标题}
    \vspace{20pt}
\end{center}

% 作者信息
\begin{center}
    \zihao{-4}\songti
    院系:XXX \quad 姓名:XXX \quad 学号:XXX
    \vspace{30pt}
\end{center}

% 摘要
\vspace{10pt}
\noindent\zihao{-4}\heiti 摘要:
\zihao{-4}\songti
在此处输入摘要内容。

\vspace{10pt}
\noindent\zihao{-4}\heiti 关键词:
\zihao{-4}\songti
关键词1;关键词2;关键词3

% 目录
\newpage
\tableofcontents
\newpage

% 正文示例
\section{研究背景与意义}
\subsection{研究背景}
在此处输入研究背景内容。

\subsection{研究意义}
在此处输入研究意义内容。

\section{文献综述}
\subsection{国内研究现状}
在此处输入国内研究现状内容。

\subsection{国外研究现状}
在此处输入国外研究现状内容。

% 表格示例
\begin{table}[H]
    \centering
    \caption{数据分析结果示例}
    \label{tab:example}
    \begin{tabular}{lccc}
        \toprule
        指标 & 2021年 & 2022年 & 2023年 \\
        \midrule
        GDP增长率(\%) & 8.1 & 3.0 & 5.2 \\
        通货膨胀率(\%) & 2.1 & 2.8 & 2.3 \\
        失业率(\%) & 5.1 & 5.5 & 5.2 \\
        \bottomrule
    \end{tabular}
    \source{数据来源:国家统计局}
\end{table}

% 图表示例
\begin{figure}[H]
    \centering
    \begin{tikzpicture}
    \begin{axis}[
        width=0.8\textwidth,
        height=0.4\textwidth,
        xlabel={年份},
        ylabel={增长率 (\%)},
        grid=major,
        legend pos=north east,
        legend style={font=\small}
    ]
    \addplot[blue,mark=*] coordinates {
        (2019,6.1)
        (2020,2.3)
        (2021,8.1)
        (2022,3.0)
        (2023,5.2)
    };
    \legend{GDP增长率}
    \end{axis}
    \end{tikzpicture}
    \caption{GDP增长率变化趋势(2019-2023)}
    \label{fig:gdp-growth}
    \source{数据来源:国家统计局}
\end{figure}

% 参考文献
\begin{thebibliography}{99}
\zihao{-4}\songti

% 中文文献
\subsection*{中文文献}

\subsubsection*{中文专著}
\bibitem{key1} 作者.书名[M].出版社,年份.

\subsubsection*{中文期刊论文}
\bibitem{key2} 作者.题目[J].期刊名,年份,卷(期):起页-止页.

% 英文文献
\subsection*{英文文献}

\subsubsection*{英文期刊论文}
\bibitem{key3} Author. Title[J]. Journal, Year.

\end{thebibliography}

\end{document}